% Options for packages loaded elsewhere
\PassOptionsToPackage{unicode}{hyperref}
\PassOptionsToPackage{hyphens}{url}
\PassOptionsToPackage{dvipsnames,svgnames*,x11names*}{xcolor}
%
\documentclass[
]{book}
\usepackage{lmodern}
\usepackage{amssymb,amsmath}
\usepackage{ifxetex,ifluatex}
\ifnum 0\ifxetex 1\fi\ifluatex 1\fi=0 % if pdftex
  \usepackage[T1]{fontenc}
  \usepackage[utf8]{inputenc}
  \usepackage{textcomp} % provide euro and other symbols
\else % if luatex or xetex
  \usepackage{unicode-math}
  \defaultfontfeatures{Scale=MatchLowercase}
  \defaultfontfeatures[\rmfamily]{Ligatures=TeX,Scale=1}
\fi
% Use upquote if available, for straight quotes in verbatim environments
\IfFileExists{upquote.sty}{\usepackage{upquote}}{}
\IfFileExists{microtype.sty}{% use microtype if available
  \usepackage[]{microtype}
  \UseMicrotypeSet[protrusion]{basicmath} % disable protrusion for tt fonts
}{}
\makeatletter
\@ifundefined{KOMAClassName}{% if non-KOMA class
  \IfFileExists{parskip.sty}{%
    \usepackage{parskip}
  }{% else
    \setlength{\parindent}{0pt}
    \setlength{\parskip}{6pt plus 2pt minus 1pt}}
}{% if KOMA class
  \KOMAoptions{parskip=half}}
\makeatother
\usepackage{xcolor}
\IfFileExists{xurl.sty}{\usepackage{xurl}}{} % add URL line breaks if available
\IfFileExists{bookmark.sty}{\usepackage{bookmark}}{\usepackage{hyperref}}
\hypersetup{
  pdftitle={A Collection of Python Examples},
  pdfauthor={Fan Wang},
  colorlinks=true,
  linkcolor=Maroon,
  filecolor=Maroon,
  citecolor=Blue,
  urlcolor=blue,
  pdfcreator={LaTeX via pandoc}}
\urlstyle{same} % disable monospaced font for URLs
\usepackage{color}
\usepackage{fancyvrb}
\newcommand{\VerbBar}{|}
\newcommand{\VERB}{\Verb[commandchars=\\\{\}]}
\DefineVerbatimEnvironment{Highlighting}{Verbatim}{commandchars=\\\{\}}
% Add ',fontsize=\small' for more characters per line
\usepackage{framed}
\definecolor{shadecolor}{RGB}{248,248,248}
\newenvironment{Shaded}{\begin{snugshade}}{\end{snugshade}}
\newcommand{\AlertTok}[1]{\textcolor[rgb]{0.94,0.16,0.16}{#1}}
\newcommand{\AnnotationTok}[1]{\textcolor[rgb]{0.56,0.35,0.01}{\textbf{\textit{#1}}}}
\newcommand{\AttributeTok}[1]{\textcolor[rgb]{0.77,0.63,0.00}{#1}}
\newcommand{\BaseNTok}[1]{\textcolor[rgb]{0.00,0.00,0.81}{#1}}
\newcommand{\BuiltInTok}[1]{#1}
\newcommand{\CharTok}[1]{\textcolor[rgb]{0.31,0.60,0.02}{#1}}
\newcommand{\CommentTok}[1]{\textcolor[rgb]{0.56,0.35,0.01}{\textit{#1}}}
\newcommand{\CommentVarTok}[1]{\textcolor[rgb]{0.56,0.35,0.01}{\textbf{\textit{#1}}}}
\newcommand{\ConstantTok}[1]{\textcolor[rgb]{0.00,0.00,0.00}{#1}}
\newcommand{\ControlFlowTok}[1]{\textcolor[rgb]{0.13,0.29,0.53}{\textbf{#1}}}
\newcommand{\DataTypeTok}[1]{\textcolor[rgb]{0.13,0.29,0.53}{#1}}
\newcommand{\DecValTok}[1]{\textcolor[rgb]{0.00,0.00,0.81}{#1}}
\newcommand{\DocumentationTok}[1]{\textcolor[rgb]{0.56,0.35,0.01}{\textbf{\textit{#1}}}}
\newcommand{\ErrorTok}[1]{\textcolor[rgb]{0.64,0.00,0.00}{\textbf{#1}}}
\newcommand{\ExtensionTok}[1]{#1}
\newcommand{\FloatTok}[1]{\textcolor[rgb]{0.00,0.00,0.81}{#1}}
\newcommand{\FunctionTok}[1]{\textcolor[rgb]{0.00,0.00,0.00}{#1}}
\newcommand{\ImportTok}[1]{#1}
\newcommand{\InformationTok}[1]{\textcolor[rgb]{0.56,0.35,0.01}{\textbf{\textit{#1}}}}
\newcommand{\KeywordTok}[1]{\textcolor[rgb]{0.13,0.29,0.53}{\textbf{#1}}}
\newcommand{\NormalTok}[1]{#1}
\newcommand{\OperatorTok}[1]{\textcolor[rgb]{0.81,0.36,0.00}{\textbf{#1}}}
\newcommand{\OtherTok}[1]{\textcolor[rgb]{0.56,0.35,0.01}{#1}}
\newcommand{\PreprocessorTok}[1]{\textcolor[rgb]{0.56,0.35,0.01}{\textit{#1}}}
\newcommand{\RegionMarkerTok}[1]{#1}
\newcommand{\SpecialCharTok}[1]{\textcolor[rgb]{0.00,0.00,0.00}{#1}}
\newcommand{\SpecialStringTok}[1]{\textcolor[rgb]{0.31,0.60,0.02}{#1}}
\newcommand{\StringTok}[1]{\textcolor[rgb]{0.31,0.60,0.02}{#1}}
\newcommand{\VariableTok}[1]{\textcolor[rgb]{0.00,0.00,0.00}{#1}}
\newcommand{\VerbatimStringTok}[1]{\textcolor[rgb]{0.31,0.60,0.02}{#1}}
\newcommand{\WarningTok}[1]{\textcolor[rgb]{0.56,0.35,0.01}{\textbf{\textit{#1}}}}
\usepackage{longtable,booktabs}
% Correct order of tables after \paragraph or \subparagraph
\usepackage{etoolbox}
\makeatletter
\patchcmd\longtable{\par}{\if@noskipsec\mbox{}\fi\par}{}{}
\makeatother
% Allow footnotes in longtable head/foot
\IfFileExists{footnotehyper.sty}{\usepackage{footnotehyper}}{\usepackage{footnote}}
\makesavenoteenv{longtable}
\usepackage{graphicx}
\makeatletter
\def\maxwidth{\ifdim\Gin@nat@width>\linewidth\linewidth\else\Gin@nat@width\fi}
\def\maxheight{\ifdim\Gin@nat@height>\textheight\textheight\else\Gin@nat@height\fi}
\makeatother
% Scale images if necessary, so that they will not overflow the page
% margins by default, and it is still possible to overwrite the defaults
% using explicit options in \includegraphics[width, height, ...]{}
\setkeys{Gin}{width=\maxwidth,height=\maxheight,keepaspectratio}
% Set default figure placement to htbp
\makeatletter
\def\fps@figure{htbp}
\makeatother
\setlength{\emergencystretch}{3em} % prevent overfull lines
\providecommand{\tightlist}{%
  \setlength{\itemsep}{0pt}\setlength{\parskip}{0pt}}
\setcounter{secnumdepth}{5}
\usepackage{bbm}
\usepackage{booktabs}
\usepackage{longtable}
\usepackage{array}
\usepackage{multirow}
\usepackage{wrapfig}
\usepackage{float}
% \floatplacement{figure}{H}
\usepackage[labelformat = empty]{caption}
\usepackage{colortbl}
\usepackage{pdflscape}
\usepackage{tabu}
\usepackage{threeparttable}
\usepackage{threeparttablex}
\usepackage[normalem]{ulem}
\usepackage{makecell}
\usepackage{xcolor}
\usepackage{geometry}
\geometry{
	a4paper,
	left=1.0in,
	right=1.0in,
	top=1.0in,
	bottom=1.0in,
}
\setcounter{secnumdepth}{5}
\setcounter{tocdepth}{5}
\usepackage[]{natbib}
\bibliographystyle{apalike}

\title{A Collection of Python Examples}
\author{Fan Wang}
\date{2020-05-09}

\begin{document}
\maketitle

{
\hypersetup{linkcolor=}
\setcounter{tocdepth}{1}
\tableofcontents
}
\hypertarget{preface}{%
\chapter*{Preface}\label{preface}}
\addcontentsline{toc}{chapter}{Preface}

This is a work-in-progress \href{https://fanwangecon.github.io/pyfan/}{website} consisting of python tutorials and examples to accomplish. Files are written with \href{https://rmarkdown.rstudio.com/}{RMD} \citep{R-rmarkdown}. Materials are gathered from various \href{https://fanwangecon.github.io/research}{projects} in which python code is used for research and paper-administrative tasks. Files are from \href{https://fanwangecon.github.io/}{\textbf{Fan}}'s \href{https://github.com/FanWangEcon/pyfan}{pyfan} repository which has an associated \href{https://pypi.org/project/pyfan/}{package}. The package functionalize various tasks tested out in the Rmd files. In addition, the \href{https://github.com/FanWangEcon/pyecon}{pyecon} repository and the associated \href{https://pypi.org/project/pyecon/}{package} contain functions and rmd files related explicitly to solving economic models. Bullet points show which python packages/functions are used to achieve various objectives.

From \href{https://fanwangecon.github.io/}{Fan}'s other repositories: For dynamic borrowing and savings problems, see \href{https://fanwangecon.github.io/CodeDynaAsset/}{Dynamic Asset Repository (Matlab)}; For code examples, see also \href{https://fanwangecon.github.io/M4Econ/}{Matlab Example Code}, \href{https://fanwangecon.github.io/R4Econ/}{R Example Code}, and \href{https://fanwangecon.github.io/Stata4Econ/}{Stata Example Code}; For intro econ with Matlab, see \href{https://fanwangecon.github.io/Math4Econ/}{Intro Mathematics for Economists}, and for intro stat with R, see \href{https://fanwangecon.github.io/Stat4Econ/}{Intro Statistics for Undergraduates}. See \href{https://github.com/FanWangEcon}{here} for all of \href{https://fanwangecon.github.io/}{Fan}'s public repositories.

The site is built using \href{https://bookdown.org/}{Bookdown} \citep{R-bookdown}.

Please contact \href{https://fanwangecon.github.io/}{FanWangEcon} for issues or problems.

\hypertarget{array-matrix-dataframe}{%
\chapter{Array, Matrix, Dataframe}\label{array-matrix-dataframe}}

\hypertarget{array}{%
\section{Array}\label{array}}

\hypertarget{strings}{%
\subsection{Strings}\label{strings}}

\begin{quote}
Go to the \href{https://github.com/FanWangEcon//pyfan/blob/master/.//A-Collection-of-Python-Examples.Rmd}{\textbf{RMD}}, \href{https://github.com/FanWangEcon//pyfan/blob/master/.//htmlpdfr/A-Collection-of-Python-Examples.pdf}{\textbf{PDF}}, or \href{https://fanwangecon.github.io//pyfan/.//htmlpdfr/A-Collection-of-Python-Examples.html}{\textbf{HTML}} version of this file. Go back to \href{http://fanwangecon.github.io/}{fan}'s \href{https://fanwangecon.github.io/pyfan/}{Python Code Examples} Repository (\href{https://fanwangecon.github.io/pyfan/bookdown}{bookdown site}).
\end{quote}

\hypertarget{replace-a-set-of-strings-in-string}{%
\subsubsection{Replace a Set of Strings in String}\label{replace-a-set-of-strings-in-string}}

Replace terms in string

\begin{Shaded}
\begin{Highlighting}[]
\CommentTok{\# define string}
\NormalTok{st\_full }\OperatorTok{=} \StringTok{"""}
\StringTok{abc is a great efg, probably xyz. Yes, xyz is great, like efg. }
\StringTok{eft good, EFG capitalized, efg good again. }
\StringTok{A B C or abc or ABC. Interesting xyz. }
\StringTok{"""}

\CommentTok{\# define new and old}
\NormalTok{ls\_st\_old }\OperatorTok{=}\NormalTok{ [}\StringTok{\textquotesingle{}abc\textquotesingle{}}\NormalTok{, }\StringTok{\textquotesingle{}efg\textquotesingle{}}\NormalTok{, }\StringTok{\textquotesingle{}xyz\textquotesingle{}}\NormalTok{]}
\NormalTok{ls\_st\_new }\OperatorTok{=}\NormalTok{ [}\StringTok{\textquotesingle{}123\textquotesingle{}}\NormalTok{, }\StringTok{\textquotesingle{}456\textquotesingle{}}\NormalTok{, }\StringTok{\textquotesingle{}789\textquotesingle{}}\NormalTok{]}

\CommentTok{\# zip and loop and replace}
\ControlFlowTok{for}\NormalTok{ old, new }\KeywordTok{in} \BuiltInTok{zip}\NormalTok{(ls\_st\_old, ls\_st\_new):}
\NormalTok{  st\_full }\OperatorTok{=}\NormalTok{ st\_full.replace(old, new)}

\CommentTok{\# print}
\BuiltInTok{print}\NormalTok{(st\_full)}
\end{Highlighting}
\end{Shaded}

\begin{verbatim}
## 
## 123 is a great 456, probably 789. Yes, 789 is great, like 456. 
## eft good, EFG capitalized, 456 good again. 
## A B C or 123 or ABC. Interesting 789.
\end{verbatim}

\hypertarget{system-and-support}{%
\chapter{System and Support}\label{system-and-support}}

\hypertarget{file-in-and-out}{%
\section{File In and Out}\label{file-in-and-out}}

\hypertarget{reading-editing-compile-and-convert-tex-with-pandoc}{%
\subsection{Reading, Editing, Compile and Convert Tex with Pandoc}\label{reading-editing-compile-and-convert-tex-with-pandoc}}

\begin{quote}
Go to the \href{https://github.com/FanWangEcon//pyfan/blob/master/.//A-Collection-of-Python-Examples.Rmd}{\textbf{RMD}}, \href{https://github.com/FanWangEcon//pyfan/blob/master/.//htmlpdfr/A-Collection-of-Python-Examples.pdf}{\textbf{PDF}}, or \href{https://fanwangecon.github.io//pyfan/.//htmlpdfr/A-Collection-of-Python-Examples.html}{\textbf{HTML}} version of this file. Go back to \href{http://fanwangecon.github.io/}{fan}'s \href{https://fanwangecon.github.io/pyfan/}{Python Code Examples} Repository (\href{https://fanwangecon.github.io/pyfan/bookdown}{bookdown site}).
\end{quote}

\begin{itemize}
\tightlist
\item
  python create a text file
\item
  python write file from paragraphs
\end{itemize}

\hypertarget{generate-a-tex-file}{%
\subsubsection{Generate a tex file}\label{generate-a-tex-file}}

Will a bare-bone tex file with some texts inside, save inside the *\_file* subfolder.

First, create the text text string, note the the linebreaks utomatically generate linebreaks, note that slash need double slash:

\begin{Shaded}
\begin{Highlighting}[]
\CommentTok{\# Create the Tex Text}
\CommentTok{\# Note that trible quotes begin first and end last lines}
\NormalTok{stf\_tex\_contents }\OperatorTok{=} \StringTok{"""}\CharTok{\textbackslash{}\textbackslash{}}\StringTok{documentclass[12pt,english]}\SpecialCharTok{\{article\}}
\CharTok{\textbackslash{}\textbackslash{}}\StringTok{usepackage[bottom]}\SpecialCharTok{\{footmisc\}}
\CharTok{\textbackslash{}\textbackslash{}}\StringTok{usepackage[urlcolor=blue]}\SpecialCharTok{\{hyperref\}}
\CharTok{\textbackslash{}\textbackslash{}}\StringTok{begin}\SpecialCharTok{\{document\}}
\CharTok{\textbackslash{}\textbackslash{}}\StringTok{title\{A Latex Testing File\}}
\CharTok{\textbackslash{}\textbackslash{}}\StringTok{author\{}\CharTok{\textbackslash{}\textbackslash{}}\StringTok{href\{http://fanwangecon.github.io/\}\{Fan Wang\} }\CharTok{\textbackslash{}\textbackslash{}}\StringTok{thanks\{See information }\CharTok{\textbackslash{}\textbackslash{}}\StringTok{href\{https://fanwangecon.github.io/Tex4Econ/\}}\SpecialCharTok{\{Tex4Econ\}}\StringTok{ for more.}\SpecialCharTok{\}\}}
\CharTok{\textbackslash{}\textbackslash{}}\StringTok{maketitle}
\StringTok{Ipsum information dolor sit amet, consectetur adipiscing elit. Integer Latex placerat nunc orci.}
\CharTok{\textbackslash{}\textbackslash{}}\StringTok{paragraph\{}\CharTok{\textbackslash{}\textbackslash{}}\StringTok{href\{https://papers.ssrn.com/sol3/papers.cfm?abstract\_id=3140132\}}\SpecialCharTok{\{Data\}}\StringTok{\}}
\StringTok{Village closure information is taken from a village head survey.}\CharTok{\textbackslash{}\textbackslash{}}\StringTok{footnote\{Generally students went to schools.\}}
\CharTok{\textbackslash{}\textbackslash{}}\StringTok{end}\SpecialCharTok{\{document\}}\StringTok{"""}
\CommentTok{\# Print}
\BuiltInTok{print}\NormalTok{(stf\_tex\_contents)}
\end{Highlighting}
\end{Shaded}

\begin{verbatim}
## \documentclass[12pt,english]{article}
## \usepackage[bottom]{footmisc}
## \usepackage[urlcolor=blue]{hyperref}
## \begin{document}
## \title{A Latex Testing File}
## \author{\href{http://fanwangecon.github.io/}{Fan Wang} \thanks{See information \href{https://fanwangecon.github.io/Tex4Econ/}{Tex4Econ} for more.}}
## \maketitle
## Ipsum information dolor sit amet, consectetur adipiscing elit. Integer Latex placerat nunc orci.
## \paragraph{\href{https://papers.ssrn.com/sol3/papers.cfm?abstract_id=3140132}{Data}}
## Village closure information is taken from a village head survey.\footnote{Generally students went to schools.}
## \end{document}
\end{verbatim}

Second, write the contents of the file to a new tex file stored inside the *\_file* subfolder of the directory:

\begin{Shaded}
\begin{Highlighting}[]
\CommentTok{\# Relative file name}
\NormalTok{srt\_file\_tex }\OperatorTok{=} \StringTok{"\_file/"}
\NormalTok{sna\_file\_tex }\OperatorTok{=} \StringTok{"test\_fan"}
\NormalTok{srn\_file\_tex }\OperatorTok{=}\NormalTok{ srt\_file\_tex }\OperatorTok{+}\NormalTok{ sna\_file\_tex }\OperatorTok{+} \StringTok{".tex"}
\CommentTok{\# Open new file}
\NormalTok{fl\_tex\_contents }\OperatorTok{=} \BuiltInTok{open}\NormalTok{(srn\_file\_tex, }\StringTok{\textquotesingle{}w\textquotesingle{}}\NormalTok{)}
\CommentTok{\# Write to File}
\NormalTok{fl\_tex\_contents.write(stf\_tex\_contents)}
\CommentTok{\# print}
\end{Highlighting}
\end{Shaded}

\begin{verbatim}
## 617
\end{verbatim}

\begin{Shaded}
\begin{Highlighting}[]
\NormalTok{fl\_tex\_contents.close()}
\end{Highlighting}
\end{Shaded}

\hypertarget{replace-strings-in-a-tex-file}{%
\subsubsection{Replace Strings in a tex file}\label{replace-strings-in-a-tex-file}}

Replace a set of strings in the file just generated by a set of alternative strings.

\begin{Shaded}
\begin{Highlighting}[]
\CommentTok{\# Open file Get text}
\NormalTok{fl\_tex\_contents }\OperatorTok{=} \BuiltInTok{open}\NormalTok{(srn\_file\_tex)}
\NormalTok{stf\_tex\_contents }\OperatorTok{=}\NormalTok{ fl\_tex\_contents.read()}
\BuiltInTok{print}\NormalTok{(srn\_file\_tex)}

\CommentTok{\# define new and old}
\end{Highlighting}
\end{Shaded}

\begin{verbatim}
## _file/test_fan.tex
\end{verbatim}

\begin{Shaded}
\begin{Highlighting}[]
\NormalTok{ls\_st\_old }\OperatorTok{=}\NormalTok{ [}\StringTok{\textquotesingle{}information\textquotesingle{}}\NormalTok{, }\StringTok{\textquotesingle{}Latex\textquotesingle{}}\NormalTok{]}
\NormalTok{ls\_st\_new }\OperatorTok{=}\NormalTok{ [}\StringTok{\textquotesingle{}INFOREPLACE\textquotesingle{}}\NormalTok{, }\StringTok{\textquotesingle{}LATEX\textquotesingle{}}\NormalTok{]}

\CommentTok{\# zip and loop and replace}
\ControlFlowTok{for}\NormalTok{ old, new }\KeywordTok{in} \BuiltInTok{zip}\NormalTok{(ls\_st\_old, ls\_st\_new):}
\NormalTok{  stf\_tex\_contents }\OperatorTok{=}\NormalTok{ stf\_tex\_contents.replace(old, new)}
\BuiltInTok{print}\NormalTok{(stf\_tex\_contents)}

\CommentTok{\# write to file with replacements }
\end{Highlighting}
\end{Shaded}

\begin{verbatim}
## \documentclass[12pt,english]{article}
## \usepackage[bottom]{footmisc}
## \usepackage[urlcolor=blue]{hyperref}
## \begin{document}
## \title{A LATEX Testing File}
## \author{\href{http://fanwangecon.github.io/}{Fan Wang} \thanks{See INFOREPLACE \href{https://fanwangecon.github.io/Tex4Econ/}{Tex4Econ} for more.}}
## \maketitle
## Ipsum INFOREPLACE dolor sit amet, consectetur adipiscing elit. Integer LATEX placerat nunc orci.
## \paragraph{\href{https://papers.ssrn.com/sol3/papers.cfm?abstract_id=3140132}{Data}}
## Village closure INFOREPLACE is taken from a village head survey.\footnote{Generally students went to schools.}
## \end{document}
\end{verbatim}

\begin{Shaded}
\begin{Highlighting}[]
\NormalTok{sna\_file\_edited\_tex }\OperatorTok{=} \StringTok{"test\_fan\_edited"}
\NormalTok{srn\_file\_edited\_tex }\OperatorTok{=}\NormalTok{ srt\_file\_tex }\OperatorTok{+}\NormalTok{ sna\_file\_edited\_tex }\OperatorTok{+} \StringTok{".tex"}
\NormalTok{fl\_tex\_ed\_contents }\OperatorTok{=} \BuiltInTok{open}\NormalTok{(srn\_file\_edited\_tex, }\StringTok{\textquotesingle{}w\textquotesingle{}}\NormalTok{)}
\NormalTok{fl\_tex\_ed\_contents.write(stf\_tex\_contents)}
\end{Highlighting}
\end{Shaded}

\begin{verbatim}
## 617
\end{verbatim}

\begin{Shaded}
\begin{Highlighting}[]
\NormalTok{fl\_tex\_ed\_contents.close()}
\end{Highlighting}
\end{Shaded}

\hypertarget{convert-tex-file-to-pandoc-and-compile-latex}{%
\subsubsection{Convert Tex File to Pandoc and Compile Latex}\label{convert-tex-file-to-pandoc-and-compile-latex}}

Compile tex file to pdf and clean up the extraneous pdf outputs.

\begin{Shaded}
\begin{Highlighting}[]
\ImportTok{import}\NormalTok{ subprocess}
\ImportTok{import}\NormalTok{ os }

\CommentTok{\# Convert tex to pdf}
\NormalTok{subprocess.call([}\StringTok{\textquotesingle{}C:/texlive/2019/bin/win32/xelatex.exe\textquotesingle{}}\NormalTok{, }\StringTok{\textquotesingle{}{-}output{-}directory\textquotesingle{}}\NormalTok{,}
\NormalTok{                 srt\_file\_tex, srn\_file\_edited\_tex], shell}\OperatorTok{=}\VariableTok{False}\NormalTok{)}
\CommentTok{\# Clean pdf extraneous output                 }
\end{Highlighting}
\end{Shaded}

\begin{verbatim}
## 0
\end{verbatim}

\begin{Shaded}
\begin{Highlighting}[]
\NormalTok{ls\_st\_remove\_suffix }\OperatorTok{=}\NormalTok{ [}\StringTok{\textquotesingle{}aux\textquotesingle{}}\NormalTok{,}\StringTok{\textquotesingle{}log\textquotesingle{}}\NormalTok{,}\StringTok{\textquotesingle{}out\textquotesingle{}}\NormalTok{,}\StringTok{\textquotesingle{}bbl\textquotesingle{}}\NormalTok{,}\StringTok{\textquotesingle{}blg\textquotesingle{}}\NormalTok{]}
\ControlFlowTok{for}\NormalTok{ st\_suffix }\KeywordTok{in}\NormalTok{ ls\_st\_remove\_suffix:}
\NormalTok{    srn\_cur\_file }\OperatorTok{=}\NormalTok{ srt\_file\_tex }\OperatorTok{+}\NormalTok{ sna\_file\_edited\_tex }\OperatorTok{+} \StringTok{"."} \OperatorTok{+}\NormalTok{ st\_suffix}
    \ControlFlowTok{if}\NormalTok{ (os.path.isfile(srn\_cur\_file)):}
\NormalTok{        os.remove(srt\_file\_tex }\OperatorTok{+}\NormalTok{ sna\_file\_edited\_tex }\OperatorTok{+} \StringTok{"."} \OperatorTok{+}\NormalTok{ st\_suffix)}
\end{Highlighting}
\end{Shaded}

Use pandoc to convert tex file

\begin{Shaded}
\begin{Highlighting}[]
\ImportTok{import}\NormalTok{ subprocess}

\CommentTok{\# md file name}
\NormalTok{srn\_file\_md }\OperatorTok{=}\NormalTok{ srt\_file\_tex }\OperatorTok{+} \StringTok{"test\_fan\_edited.md"}
\CommentTok{\# Convert tex to md}
\NormalTok{subprocess.call([}\StringTok{\textquotesingle{}pandoc\textquotesingle{}}\NormalTok{, }\StringTok{\textquotesingle{}{-}s\textquotesingle{}}\NormalTok{, srn\_file\_tex, }\StringTok{\textquotesingle{}{-}o\textquotesingle{}}\NormalTok{, srn\_file\_md])}
\CommentTok{\# Open md file}
\end{Highlighting}
\end{Shaded}

\begin{verbatim}
## 0
\end{verbatim}

\begin{Shaded}
\begin{Highlighting}[]
\NormalTok{fl\_md\_contents }\OperatorTok{=} \BuiltInTok{open}\NormalTok{(srn\_file\_md)}
\BuiltInTok{print}\NormalTok{(fl\_md\_contents.read())}
\end{Highlighting}
\end{Shaded}

\begin{verbatim}
## ---
## author:
## - '[Fan Wang](http://fanwangecon.github.io/) [^1]'
## title: A Latex Testing File
## ---
## 
## \maketitle
## Ipsum information dolor sit amet, consectetur adipiscing elit. Integer
## Latex placerat nunc orci.
## 
## #### [Data](https://papers.ssrn.com/sol3/papers.cfm?abstract_id=3140132)
## 
## Village closure information is taken from a village head survey.[^2]
## 
## [^1]: See information
##     [Tex4Econ](https://fanwangecon.github.io/Tex4Econ/) for more.
## 
## [^2]: Generally students went to schools.
\end{verbatim}

\hypertarget{search-for-files-with-suffix-in-several-folders}{%
\subsubsection{Search for Files with Suffix in Several Folders}\label{search-for-files-with-suffix-in-several-folders}}

\begin{itemize}
\tightlist
\item
  python search all files in folders with suffix
\end{itemize}

Search for files in several directories that have a particular suffix. Then decompose directory into sub-components.

\begin{Shaded}
\begin{Highlighting}[]
\ImportTok{from}\NormalTok{ pathlib }\ImportTok{import}\NormalTok{ Path}

\CommentTok{\# directories to search in}
\NormalTok{ls\_spt\_srh }\OperatorTok{=}\NormalTok{ [}\StringTok{"C:/Users/fan/R4Econ/amto/array/"}\NormalTok{, }
              \StringTok{"C:/Users/fan/R4Econ/amto/list"}\NormalTok{]}

\CommentTok{\# get file names}
\NormalTok{ls\_spn\_found }\OperatorTok{=}\NormalTok{ [spn\_file }\ControlFlowTok{for}\NormalTok{ spt\_srh }\KeywordTok{in}\NormalTok{ ls\_spt\_srh }
                         \ControlFlowTok{for}\NormalTok{ spn\_file }\KeywordTok{in}\NormalTok{ Path(spt\_srh).rglob(}\StringTok{\textquotesingle{}*.R\textquotesingle{}}\NormalTok{)]}
\ControlFlowTok{for}\NormalTok{ spn\_found }\KeywordTok{in}\NormalTok{ ls\_spn\_found:}
\NormalTok{  drive, path\_and\_file }\OperatorTok{=}\NormalTok{ os.path.splitdrive(spn\_found)}
\NormalTok{  path\_no\_suffix }\OperatorTok{=}\NormalTok{ os.path.splitext(spn\_found)[}\DecValTok{0}\NormalTok{]}
\NormalTok{  path\_no\_file, }\BuiltInTok{file} \OperatorTok{=}\NormalTok{ os.path.split(spn\_found)}
\NormalTok{  file\_no\_suffix }\OperatorTok{=}\NormalTok{ Path(spn\_found).stem}
  \BuiltInTok{print}\NormalTok{(}\StringTok{\textquotesingle{}file:\textquotesingle{}}\NormalTok{, }\BuiltInTok{file}\NormalTok{, }\StringTok{\textquotesingle{}}\CharTok{\textbackslash{}n}\StringTok{drive:\textquotesingle{}}\NormalTok{, drive,}
        \StringTok{\textquotesingle{}}\CharTok{\textbackslash{}n}\StringTok{file no suffix:\textquotesingle{}}\NormalTok{, file\_no\_suffix,}
        \StringTok{\textquotesingle{}}\CharTok{\textbackslash{}n}\StringTok{full path:\textquotesingle{}}\NormalTok{, spn\_found, }
        \StringTok{\textquotesingle{}}\CharTok{\textbackslash{}n}\StringTok{pt no fle:\textquotesingle{}}\NormalTok{, path\_no\_file,}
        \StringTok{\textquotesingle{}}\CharTok{\textbackslash{}n}\StringTok{pt no suf:\textquotesingle{}}\NormalTok{, path\_no\_suffix, }\StringTok{\textquotesingle{}}\CharTok{\textbackslash{}n}\StringTok{\textquotesingle{}}\NormalTok{)}
\end{Highlighting}
\end{Shaded}

\begin{verbatim}
## file: fs_ary_basics.R 
## drive: C: 
## file no suffix: fs_ary_basics 
## full path: C:\Users\fan\R4Econ\amto\array\htmlpdfr\fs_ary_basics.R 
## pt no fle: C:\Users\fan\R4Econ\amto\array\htmlpdfr 
## pt no suf: C:\Users\fan\R4Econ\amto\array\htmlpdfr\fs_ary_basics 
## 
## file: fs_ary_generate.R 
## drive: C: 
## file no suffix: fs_ary_generate 
## full path: C:\Users\fan\R4Econ\amto\array\htmlpdfr\fs_ary_generate.R 
## pt no fle: C:\Users\fan\R4Econ\amto\array\htmlpdfr 
## pt no suf: C:\Users\fan\R4Econ\amto\array\htmlpdfr\fs_ary_generate 
## 
## file: fs_ary_mesh.R 
## drive: C: 
## file no suffix: fs_ary_mesh 
## full path: C:\Users\fan\R4Econ\amto\array\htmlpdfr\fs_ary_mesh.R 
## pt no fle: C:\Users\fan\R4Econ\amto\array\htmlpdfr 
## pt no suf: C:\Users\fan\R4Econ\amto\array\htmlpdfr\fs_ary_mesh 
## 
## file: fs_ary_string.R 
## drive: C: 
## file no suffix: fs_ary_string 
## full path: C:\Users\fan\R4Econ\amto\array\htmlpdfr\fs_ary_string.R 
## pt no fle: C:\Users\fan\R4Econ\amto\array\htmlpdfr 
## pt no suf: C:\Users\fan\R4Econ\amto\array\htmlpdfr\fs_ary_string 
## 
## file: fs_meshr.R 
## drive: C: 
## file no suffix: fs_meshr 
## full path: C:\Users\fan\R4Econ\amto\array\htmlpdfr\fs_meshr.R 
## pt no fle: C:\Users\fan\R4Econ\amto\array\htmlpdfr 
## pt no suf: C:\Users\fan\R4Econ\amto\array\htmlpdfr\fs_meshr 
## 
## file: fs_listr.R 
## drive: C: 
## file no suffix: fs_listr 
## full path: C:\Users\fan\R4Econ\amto\list\htmlpdfr\fs_listr.R 
## pt no fle: C:\Users\fan\R4Econ\amto\list\htmlpdfr 
## pt no suf: C:\Users\fan\R4Econ\amto\list\htmlpdfr\fs_listr 
## 
## file: fs_lst_basics.R 
## drive: C: 
## file no suffix: fs_lst_basics 
## full path: C:\Users\fan\R4Econ\amto\list\htmlpdfr\fs_lst_basics.R 
## pt no fle: C:\Users\fan\R4Econ\amto\list\htmlpdfr 
## pt no suf: C:\Users\fan\R4Econ\amto\list\htmlpdfr\fs_lst_basics
\end{verbatim}

\hypertarget{appendix-appendix}{%
\appendix}


\hypertarget{index-and-code-links}{%
\chapter{Index and Code Links}\label{index-and-code-links}}

\hypertarget{array-matrix-dataframe-links}{%
\section{Array, Matrix, Dataframe links}\label{array-matrix-dataframe-links}}

\hypertarget{section-1.1-arrayarray-links}{%
\subsection{\texorpdfstring{\protect\hyperlink{array}{Section 1.1 Array} links}{Section 1.1 Array links}}\label{section-1.1-arrayarray-links}}

\begin{enumerate}
\def\labelenumi{\arabic{enumi}.}
\tightlist
\item
  \href{https://fanwangecon.github.io/pyfan/vig/amto/array/htmlpdfr/fp_ary_string.html}{Python String Manipulation Examples}: \href{https://github.com/FanWangEcon/pyfan/blob/master/vig/amto/array//fp_ary_string.Rmd}{\textbf{rmd}} \textbar{} \href{https://github.com/FanWangEcon/pyfan/blob/master/vig/amto/array/htmlpdfr/fp_ary_string.R}{\textbf{r}} \textbar{} \href{https://github.com/FanWangEcon/pyfan/blob/master/vig/amto/array/htmlpdfr/fp_ary_string.pdf}{\textbf{pdf}} \textbar{} \href{https://fanwangecon.github.io/pyfan/vig/amto/array/htmlpdfr/fp_ary_string.html}{\textbf{html}}

  \begin{itemize}
  \tightlist
  \item
    Various string manipulations
  \item
    \textbf{py}: \emph{zip()}
  \end{itemize}
\end{enumerate}

\hypertarget{system-and-support-links}{%
\section{System and Support links}\label{system-and-support-links}}

\hypertarget{section-2.1-file-in-and-outfile-in-and-out-links}{%
\subsection{\texorpdfstring{\protect\hyperlink{file-in-and-out}{Section 2.1 File In and Out} links}{Section 2.1 File In and Out links}}\label{section-2.1-file-in-and-outfile-in-and-out-links}}

\begin{enumerate}
\def\labelenumi{\arabic{enumi}.}
\tightlist
\item
  \href{https://fanwangecon.github.io/pyfan/vig/support/inout/htmlpdfr/fp_files.html}{Python Reading and Writing to File Examples}: \href{https://github.com/FanWangEcon/pyfan/blob/master/vig/support/inout//fp_files.Rmd}{\textbf{rmd}} \textbar{} \href{https://github.com/FanWangEcon/pyfan/blob/master/vig/support/inout/htmlpdfr/fp_files.R}{\textbf{r}} \textbar{} \href{https://github.com/FanWangEcon/pyfan/blob/master/vig/support/inout/htmlpdfr/fp_files.pdf}{\textbf{pdf}} \textbar{} \href{https://fanwangecon.github.io/pyfan/vig/support/inout/htmlpdfr/fp_files.html}{\textbf{html}}

  \begin{itemize}
  \tightlist
  \item
    Reading from file and replace strings in file.
  \item
    \textbf{py}: \emph{open() + write() + replace() + {[}c for b in {[}{[}1,2{]},{[}2,3{]}{]} for c in b{]}}
  \item
    \textbf{subprocess}: \emph{read()}
  \item
    \textbf{pathlib}: \emph{Path().rglob() + Path().stem}
  \item
    \textbf{os}: \emph{remove() + listdir() + path.isfile() + path.splitdrive() + os.path.splitext() + os.path.split()}
  \end{itemize}
\end{enumerate}

  \bibliography{book.bib,packages.bib}

\end{document}
