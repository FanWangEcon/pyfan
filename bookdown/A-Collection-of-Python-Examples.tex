% Options for packages loaded elsewhere
\PassOptionsToPackage{unicode}{hyperref}
\PassOptionsToPackage{hyphens}{url}
\PassOptionsToPackage{dvipsnames,svgnames*,x11names*}{xcolor}
%
\documentclass[
]{book}
\usepackage{lmodern}
\usepackage{amssymb,amsmath}
\usepackage{ifxetex,ifluatex}
\ifnum 0\ifxetex 1\fi\ifluatex 1\fi=0 % if pdftex
  \usepackage[T1]{fontenc}
  \usepackage[utf8]{inputenc}
  \usepackage{textcomp} % provide euro and other symbols
\else % if luatex or xetex
  \usepackage{unicode-math}
  \defaultfontfeatures{Scale=MatchLowercase}
  \defaultfontfeatures[\rmfamily]{Ligatures=TeX,Scale=1}
\fi
% Use upquote if available, for straight quotes in verbatim environments
\IfFileExists{upquote.sty}{\usepackage{upquote}}{}
\IfFileExists{microtype.sty}{% use microtype if available
  \usepackage[]{microtype}
  \UseMicrotypeSet[protrusion]{basicmath} % disable protrusion for tt fonts
}{}
\makeatletter
\@ifundefined{KOMAClassName}{% if non-KOMA class
  \IfFileExists{parskip.sty}{%
    \usepackage{parskip}
  }{% else
    \setlength{\parindent}{0pt}
    \setlength{\parskip}{6pt plus 2pt minus 1pt}}
}{% if KOMA class
  \KOMAoptions{parskip=half}}
\makeatother
\usepackage{xcolor}
\IfFileExists{xurl.sty}{\usepackage{xurl}}{} % add URL line breaks if available
\IfFileExists{bookmark.sty}{\usepackage{bookmark}}{\usepackage{hyperref}}
\hypersetup{
  pdftitle={A Collection of Python Examples},
  pdfauthor={Fan Wang},
  colorlinks=true,
  linkcolor=Maroon,
  filecolor=Maroon,
  citecolor=Blue,
  urlcolor=blue,
  pdfcreator={LaTeX via pandoc}}
\urlstyle{same} % disable monospaced font for URLs
\usepackage{color}
\usepackage{fancyvrb}
\newcommand{\VerbBar}{|}
\newcommand{\VERB}{\Verb[commandchars=\\\{\}]}
\DefineVerbatimEnvironment{Highlighting}{Verbatim}{commandchars=\\\{\}}
% Add ',fontsize=\small' for more characters per line
\usepackage{framed}
\definecolor{shadecolor}{RGB}{248,248,248}
\newenvironment{Shaded}{\begin{snugshade}}{\end{snugshade}}
\newcommand{\AlertTok}[1]{\textcolor[rgb]{0.94,0.16,0.16}{#1}}
\newcommand{\AnnotationTok}[1]{\textcolor[rgb]{0.56,0.35,0.01}{\textbf{\textit{#1}}}}
\newcommand{\AttributeTok}[1]{\textcolor[rgb]{0.77,0.63,0.00}{#1}}
\newcommand{\BaseNTok}[1]{\textcolor[rgb]{0.00,0.00,0.81}{#1}}
\newcommand{\BuiltInTok}[1]{#1}
\newcommand{\CharTok}[1]{\textcolor[rgb]{0.31,0.60,0.02}{#1}}
\newcommand{\CommentTok}[1]{\textcolor[rgb]{0.56,0.35,0.01}{\textit{#1}}}
\newcommand{\CommentVarTok}[1]{\textcolor[rgb]{0.56,0.35,0.01}{\textbf{\textit{#1}}}}
\newcommand{\ConstantTok}[1]{\textcolor[rgb]{0.00,0.00,0.00}{#1}}
\newcommand{\ControlFlowTok}[1]{\textcolor[rgb]{0.13,0.29,0.53}{\textbf{#1}}}
\newcommand{\DataTypeTok}[1]{\textcolor[rgb]{0.13,0.29,0.53}{#1}}
\newcommand{\DecValTok}[1]{\textcolor[rgb]{0.00,0.00,0.81}{#1}}
\newcommand{\DocumentationTok}[1]{\textcolor[rgb]{0.56,0.35,0.01}{\textbf{\textit{#1}}}}
\newcommand{\ErrorTok}[1]{\textcolor[rgb]{0.64,0.00,0.00}{\textbf{#1}}}
\newcommand{\ExtensionTok}[1]{#1}
\newcommand{\FloatTok}[1]{\textcolor[rgb]{0.00,0.00,0.81}{#1}}
\newcommand{\FunctionTok}[1]{\textcolor[rgb]{0.00,0.00,0.00}{#1}}
\newcommand{\ImportTok}[1]{#1}
\newcommand{\InformationTok}[1]{\textcolor[rgb]{0.56,0.35,0.01}{\textbf{\textit{#1}}}}
\newcommand{\KeywordTok}[1]{\textcolor[rgb]{0.13,0.29,0.53}{\textbf{#1}}}
\newcommand{\NormalTok}[1]{#1}
\newcommand{\OperatorTok}[1]{\textcolor[rgb]{0.81,0.36,0.00}{\textbf{#1}}}
\newcommand{\OtherTok}[1]{\textcolor[rgb]{0.56,0.35,0.01}{#1}}
\newcommand{\PreprocessorTok}[1]{\textcolor[rgb]{0.56,0.35,0.01}{\textit{#1}}}
\newcommand{\RegionMarkerTok}[1]{#1}
\newcommand{\SpecialCharTok}[1]{\textcolor[rgb]{0.00,0.00,0.00}{#1}}
\newcommand{\SpecialStringTok}[1]{\textcolor[rgb]{0.31,0.60,0.02}{#1}}
\newcommand{\StringTok}[1]{\textcolor[rgb]{0.31,0.60,0.02}{#1}}
\newcommand{\VariableTok}[1]{\textcolor[rgb]{0.00,0.00,0.00}{#1}}
\newcommand{\VerbatimStringTok}[1]{\textcolor[rgb]{0.31,0.60,0.02}{#1}}
\newcommand{\WarningTok}[1]{\textcolor[rgb]{0.56,0.35,0.01}{\textbf{\textit{#1}}}}
\usepackage{longtable,booktabs}
% Correct order of tables after \paragraph or \subparagraph
\usepackage{etoolbox}
\makeatletter
\patchcmd\longtable{\par}{\if@noskipsec\mbox{}\fi\par}{}{}
\makeatother
% Allow footnotes in longtable head/foot
\IfFileExists{footnotehyper.sty}{\usepackage{footnotehyper}}{\usepackage{footnote}}
\makesavenoteenv{longtable}
\usepackage{graphicx,grffile}
\makeatletter
\def\maxwidth{\ifdim\Gin@nat@width>\linewidth\linewidth\else\Gin@nat@width\fi}
\def\maxheight{\ifdim\Gin@nat@height>\textheight\textheight\else\Gin@nat@height\fi}
\makeatother
% Scale images if necessary, so that they will not overflow the page
% margins by default, and it is still possible to overwrite the defaults
% using explicit options in \includegraphics[width, height, ...]{}
\setkeys{Gin}{width=\maxwidth,height=\maxheight,keepaspectratio}
% Set default figure placement to htbp
\makeatletter
\def\fps@figure{htbp}
\makeatother
\setlength{\emergencystretch}{3em} % prevent overfull lines
\providecommand{\tightlist}{%
  \setlength{\itemsep}{0pt}\setlength{\parskip}{0pt}}
\setcounter{secnumdepth}{5}
\usepackage{bbm}
\usepackage{booktabs}
\usepackage{longtable}
\usepackage{array}
\usepackage{multirow}
\usepackage{wrapfig}
\usepackage{float}
% \floatplacement{figure}{H}
\usepackage[labelformat = empty]{caption}
\usepackage{colortbl}
\usepackage{pdflscape}
\usepackage{tabu}
\usepackage{threeparttable}
\usepackage{threeparttablex}
\usepackage[normalem]{ulem}
\usepackage{makecell}
\usepackage{xcolor}
\usepackage{geometry}
\geometry{
	a4paper,
	left=1.0in,
	right=1.0in,
	top=1.0in,
	bottom=1.0in,
}
\setcounter{secnumdepth}{5}
\setcounter{tocdepth}{5}
\usepackage[]{natbib}
\bibliographystyle{apalike}

\title{A Collection of Python Examples}
\author{Fan Wang}
\date{2020-05-24}

\begin{document}
\maketitle

{
\hypersetup{linkcolor=}
\setcounter{tocdepth}{2}
\tableofcontents
}
\hypertarget{preface}{%
\chapter*{Preface}\label{preface}}
\addcontentsline{toc}{chapter}{Preface}

This is a work-in-progress \href{https://fanwangecon.github.io/pyfan/}{website} consisting of python tutorials and examples to accomplish. Files are written with \href{https://rmarkdown.rstudio.com/}{RMD} \citep{R-rmarkdown}. Materials are gathered from various \href{https://fanwangecon.github.io/research}{projects} in which python code is used for research and paper-administrative tasks. Files are from \href{https://fanwangecon.github.io/}{\textbf{Fan}}'s \href{https://github.com/FanWangEcon/pyfan}{pyfan} repository which has an associated \href{https://pypi.org/project/pyfan/}{package}. The package functionalize various tasks tested out in the Rmd files. In addition, the \href{https://github.com/FanWangEcon/pyecon}{pyecon} repository and the associated \href{https://pypi.org/project/pyecon/}{package} (\href{https://pyfan.readthedocs.io/en/latest/autoapi/pyfan/index.html\#module-pyfan}{readthedocs}) contain functions and rmd files related explicitly to solving economic models.

From \href{https://fanwangecon.github.io/}{Fan}'s other repositories: For dynamic borrowing and savings problems, see \href{https://fanwangecon.github.io/CodeDynaAsset/}{Dynamic Asset Repository (Matlab)}; For code examples, see also \href{https://fanwangecon.github.io/M4Econ/}{Matlab Example Code}, \href{https://fanwangecon.github.io/R4Econ/}{R Example Code}, and \href{https://fanwangecon.github.io/Stata4Econ/}{Stata Example Code}; For intro econ with Matlab, see \href{https://fanwangecon.github.io/Math4Econ/}{Intro Mathematics for Economists}, and for intro stat with R, see \href{https://fanwangecon.github.io/Stat4Econ/}{Intro Statistics for Undergraduates}. See \href{https://github.com/FanWangEcon}{here} for all of \href{https://fanwangecon.github.io/}{Fan}'s public repositories.

The site is built using \href{https://bookdown.org/}{Bookdown} \citep{R-bookdown}.

Please contact \href{https://fanwangecon.github.io/}{FanWangEcon} for issues or problems.

\hypertarget{array-matrix-dataframe}{%
\chapter{Array, Matrix, Dataframe}\label{array-matrix-dataframe}}

\hypertarget{array}{%
\section{Array}\label{array}}

\hypertarget{strings}{%
\subsection{Strings}\label{strings}}

\begin{quote}
Go to the \href{https://github.com/FanWangEcon//pyfan/blob/master/.//A-Collection-of-Python-Examples.Rmd}{\textbf{RMD}}, \href{https://github.com/FanWangEcon//pyfan/blob/master/.//htmlpdfr/A-Collection-of-Python-Examples.pdf}{\textbf{PDF}}, or \href{https://fanwangecon.github.io//pyfan/.//htmlpdfr/A-Collection-of-Python-Examples.html}{\textbf{HTML}} version of this file. Go back to \href{http://fanwangecon.github.io/}{fan}'s \href{https://fanwangecon.github.io/pyfan/}{Python Code Examples} Repository (\href{https://fanwangecon.github.io/pyfan/bookdown}{bookdown site}).
\end{quote}

\hypertarget{search-if-names-include-strings}{%
\subsubsection{Search if Names Include Strings}\label{search-if-names-include-strings}}

Given a list of strings, loop but skip if string contains elements string list.

\begin{Shaded}
\begin{Highlighting}[]
\CommentTok{# define string}
\NormalTok{ls_st_ignore }\OperatorTok{=}\NormalTok{ [}\StringTok{'abc'}\NormalTok{, }\StringTok{'efg'}\NormalTok{, }\StringTok{'xyz'}\NormalTok{]}
\NormalTok{ls_st_loop }\OperatorTok{=}\NormalTok{ [}\StringTok{'ab cefg sdf'}\NormalTok{, }\StringTok{'12345'}\NormalTok{, }\StringTok{'xyz'}\NormalTok{, }\StringTok{'abc xyz'}\NormalTok{, }\StringTok{'good morning'}\NormalTok{]}

\CommentTok{# zip and loop and replace}
\ControlFlowTok{for}\NormalTok{ st_loop }\KeywordTok{in}\NormalTok{ ls_st_loop:}
  \ControlFlowTok{if} \BuiltInTok{sum}\NormalTok{([st_ignore }\KeywordTok{in}\NormalTok{ st_loop }\ControlFlowTok{for}\NormalTok{ st_ignore }\KeywordTok{in}\NormalTok{ ls_st_ignore]):}
    \BuiltInTok{print}\NormalTok{(}\StringTok{'skip:'}\NormalTok{, st_loop)}
  \ControlFlowTok{else}\NormalTok{:}
    \BuiltInTok{print}\NormalTok{(}\StringTok{'not skip:'}\NormalTok{, st_loop)}
\end{Highlighting}
\end{Shaded}

\begin{verbatim}
## skip: ab cefg sdf
## not skip: 12345
## skip: xyz
## skip: abc xyz
## not skip: good morning
\end{verbatim}

\hypertarget{replace-a-set-of-strings-in-string}{%
\subsubsection{Replace a Set of Strings in String}\label{replace-a-set-of-strings-in-string}}

Replace terms in string

\begin{Shaded}
\begin{Highlighting}[]
\CommentTok{# define string}
\NormalTok{st_full }\OperatorTok{=} \StringTok{"""}
\StringTok{abc is a great efg, probably xyz. Yes, xyz is great, like efg. }
\StringTok{eft good, EFG capitalized, efg good again. }
\StringTok{A B C or abc or ABC. Interesting xyz. }
\StringTok{"""}

\CommentTok{# define new and old}
\NormalTok{ls_st_old }\OperatorTok{=}\NormalTok{ [}\StringTok{'abc'}\NormalTok{, }\StringTok{'efg'}\NormalTok{, }\StringTok{'xyz'}\NormalTok{]}
\NormalTok{ls_st_new }\OperatorTok{=}\NormalTok{ [}\StringTok{'123'}\NormalTok{, }\StringTok{'456'}\NormalTok{, }\StringTok{'789'}\NormalTok{]}

\CommentTok{# zip and loop and replace}
\ControlFlowTok{for}\NormalTok{ old, new }\KeywordTok{in} \BuiltInTok{zip}\NormalTok{(ls_st_old, ls_st_new):}
\NormalTok{  st_full }\OperatorTok{=}\NormalTok{ st_full.replace(old, new)}

\CommentTok{# print}
\BuiltInTok{print}\NormalTok{(st_full)}
\end{Highlighting}
\end{Shaded}

\begin{verbatim}
## 
## 123 is a great 456, probably 789. Yes, 789 is great, like 456. 
## eft good, EFG capitalized, 456 good again. 
## A B C or 123 or ABC. Interesting 789.
\end{verbatim}

\hypertarget{dictionary}{%
\section{Dictionary}\label{dictionary}}

\hypertarget{dictionary-1}{%
\subsection{Dictionary}\label{dictionary-1}}

\begin{quote}
Go to the \href{https://github.com/FanWangEcon//pyfan/blob/master/.//A-Collection-of-Python-Examples.Rmd}{\textbf{RMD}}, \href{https://github.com/FanWangEcon//pyfan/blob/master/.//htmlpdfr/A-Collection-of-Python-Examples.pdf}{\textbf{PDF}}, or \href{https://fanwangecon.github.io//pyfan/.//htmlpdfr/A-Collection-of-Python-Examples.html}{\textbf{HTML}} version of this file. Go back to \href{http://fanwangecon.github.io/}{fan}'s \href{https://fanwangecon.github.io/pyfan/}{Python Code Examples} Repository (\href{https://fanwangecon.github.io/pyfan/bookdown}{bookdown site}).
\end{quote}

\hypertarget{create-a-list-of-dictionaries}{%
\subsubsection{Create a List of Dictionaries}\label{create-a-list-of-dictionaries}}

\begin{Shaded}
\begin{Highlighting}[]
\ImportTok{import}\NormalTok{ datetime}
\ImportTok{import}\NormalTok{ pprint}
\NormalTok{ls_dc_exa }\OperatorTok{=}\NormalTok{  [}
\NormalTok{    \{}\StringTok{"file"}\NormalTok{: }\StringTok{"mat_matlab"}\NormalTok{,}
     \StringTok{"title"}\NormalTok{: }\StringTok{"One Variable Graphs and Tables"}\NormalTok{,}
     \StringTok{"description"}\NormalTok{: }\StringTok{"Frequency table, bar chart and histogram"}\NormalTok{,}
     \StringTok{"val"}\NormalTok{: }\DecValTok{1}\NormalTok{,}
     \StringTok{"date"}\NormalTok{: datetime.date(}\DecValTok{2020}\NormalTok{, }\DecValTok{5}\NormalTok{, }\DecValTok{2}\NormalTok{)\},}
\NormalTok{    \{}\StringTok{"file"}\NormalTok{: }\StringTok{"mat_two"}\NormalTok{,}
     \StringTok{"title"}\NormalTok{: }\StringTok{"Second file"}\NormalTok{,}
     \StringTok{"description"}\NormalTok{: }\StringTok{"Second file."}\NormalTok{,}
     \StringTok{"val"}\NormalTok{: [}\DecValTok{1}\NormalTok{, }\DecValTok{2}\NormalTok{, }\DecValTok{3}\NormalTok{],}
     \StringTok{"date"}\NormalTok{: datetime.date(}\DecValTok{2020}\NormalTok{, }\DecValTok{5}\NormalTok{, }\DecValTok{2}\NormalTok{)\},}
\NormalTok{    \{}\StringTok{"file"}\NormalTok{: }\StringTok{"mat_algebra_rules"}\NormalTok{,}
     \StringTok{"title"}\NormalTok{: }\StringTok{"Opening a Dataset"}\NormalTok{,}
     \StringTok{"description"}\NormalTok{: }\StringTok{"Opening a Dataset."}\NormalTok{,}
     \StringTok{"val"}\NormalTok{: }\FloatTok{1.1}\NormalTok{,}
     \StringTok{"date"}\NormalTok{: datetime.date(}\DecValTok{2018}\NormalTok{, }\DecValTok{12}\NormalTok{, }\DecValTok{1}\NormalTok{)\}}
\NormalTok{]}
\NormalTok{pprint.pprint(ls_dc_exa, width}\OperatorTok{=}\DecValTok{1}\NormalTok{)}
\end{Highlighting}
\end{Shaded}

\begin{verbatim}
## [{'date': datetime.date(2020, 5, 2),
##   'description': 'Frequency '
##                  'table, '
##                  'bar '
##                  'chart '
##                  'and '
##                  'histogram',
##   'file': 'mat_matlab',
##   'title': 'One '
##            'Variable '
##            'Graphs '
##            'and '
##            'Tables',
##   'val': 1},
##  {'date': datetime.date(2020, 5, 2),
##   'description': 'Second '
##                  'file.',
##   'file': 'mat_two',
##   'title': 'Second '
##            'file',
##   'val': [1,
##           2,
##           3]},
##  {'date': datetime.date(2018, 12, 1),
##   'description': 'Opening '
##                  'a '
##                  'Dataset.',
##   'file': 'mat_algebra_rules',
##   'title': 'Opening '
##            'a '
##            'Dataset',
##   'val': 1.1}]
\end{verbatim}

\hypertarget{select-by-keys-in-dictionary}{%
\subsubsection{Select by Keys in Dictionary}\label{select-by-keys-in-dictionary}}

Given a list of dictionary, search if key name is in list:

\begin{Shaded}
\begin{Highlighting}[]
\CommentTok{# string to search through}
\NormalTok{ls_str_file_ids }\OperatorTok{=}\NormalTok{ [}\StringTok{'mat_matlab'}\NormalTok{, }\StringTok{'mat_algebra_rules'}\NormalTok{]}
\CommentTok{# select subset}
\NormalTok{ls_dc_selected }\OperatorTok{=}\NormalTok{ [dc_exa}
                  \ControlFlowTok{for}\NormalTok{ dc_exa }\KeywordTok{in}\NormalTok{ ls_dc_exa}
                  \ControlFlowTok{if}\NormalTok{ dc_exa[}\StringTok{'file'}\NormalTok{] }\KeywordTok{in}\NormalTok{ ls_str_file_ids]}
\CommentTok{# print}
\NormalTok{pprint.pprint(ls_dc_selected, width}\OperatorTok{=}\DecValTok{1}\NormalTok{)}
\end{Highlighting}
\end{Shaded}

\begin{verbatim}
## [{'date': datetime.date(2020, 5, 2),
##   'description': 'Frequency '
##                  'table, '
##                  'bar '
##                  'chart '
##                  'and '
##                  'histogram',
##   'file': 'mat_matlab',
##   'title': 'One '
##            'Variable '
##            'Graphs '
##            'and '
##            'Tables',
##   'val': 1},
##  {'date': datetime.date(2018, 12, 1),
##   'description': 'Opening '
##                  'a '
##                  'Dataset.',
##   'file': 'mat_algebra_rules',
##   'title': 'Opening '
##            'a '
##            'Dataset',
##   'val': 1.1}]
\end{verbatim}

Search and Select by Multiple Keys in Dictionary. Using two keys below:

\begin{Shaded}
\begin{Highlighting}[]
\CommentTok{# string to search through}
\NormalTok{ls_str_file_ids }\OperatorTok{=}\NormalTok{ [}\StringTok{'mat_matlab'}\NormalTok{, }\StringTok{'mat_algebra_rules'}\NormalTok{]}
\CommentTok{# select subset}
\NormalTok{ls_dc_selected }\OperatorTok{=}\NormalTok{ [dc_exa}
                  \ControlFlowTok{for}\NormalTok{ dc_exa }\KeywordTok{in}\NormalTok{ ls_dc_exa}
                  \ControlFlowTok{if}\NormalTok{ ((dc_exa[}\StringTok{'file'}\NormalTok{] }\KeywordTok{in}\NormalTok{ ls_str_file_ids) }
                      \KeywordTok{and}
\NormalTok{                      (dc_exa[}\StringTok{'val'}\NormalTok{]}\OperatorTok{==} \DecValTok{1}\NormalTok{))]}
\CommentTok{# print}
\NormalTok{pprint.pprint(ls_dc_selected, width}\OperatorTok{=}\DecValTok{1}\NormalTok{)}
\end{Highlighting}
\end{Shaded}

\begin{verbatim}
## [{'date': datetime.date(2020, 5, 2),
##   'description': 'Frequency '
##                  'table, '
##                  'bar '
##                  'chart '
##                  'and '
##                  'histogram',
##   'file': 'mat_matlab',
##   'title': 'One '
##            'Variable '
##            'Graphs '
##            'and '
##            'Tables',
##   'val': 1}]
\end{verbatim}

\hypertarget{system-and-support}{%
\chapter{System and Support}\label{system-and-support}}

\hypertarget{file-in-and-out}{%
\section{File In and Out}\label{file-in-and-out}}

\hypertarget{read-and-write-and-convert}{%
\subsection{Read and Write and Convert}\label{read-and-write-and-convert}}

\begin{quote}
Go to the \href{https://github.com/FanWangEcon//pyfan/blob/master/.//A-Collection-of-Python-Examples.Rmd}{\textbf{RMD}}, \href{https://github.com/FanWangEcon//pyfan/blob/master/.//htmlpdfr/A-Collection-of-Python-Examples.pdf}{\textbf{PDF}}, or \href{https://fanwangecon.github.io//pyfan/.//htmlpdfr/A-Collection-of-Python-Examples.html}{\textbf{HTML}} version of this file. Go back to \href{http://fanwangecon.github.io/}{fan}'s \href{https://fanwangecon.github.io/pyfan/}{Python Code Examples} Repository (\href{https://fanwangecon.github.io/pyfan/bookdown}{bookdown site}).
\end{quote}

\begin{itemize}
\tightlist
\item
  python create a text file
\item
  python write file from paragraphs
\end{itemize}

\hypertarget{generate-a-tex-file}{%
\subsubsection{Generate a tex file}\label{generate-a-tex-file}}

Will a bare-bone tex file with some texts inside, save inside the *\_file* subfolder.

First, create the text text string, note the the linebreaks utomatically generate linebreaks, note that slash need double slash:

\begin{Shaded}
\begin{Highlighting}[]
\CommentTok{# Create the Tex Text}
\CommentTok{# Note that trible quotes begin first and end last lines}
\NormalTok{stf_tex_contents }\OperatorTok{=} \StringTok{"""}\CharTok{\textbackslash{}\textbackslash{}}\StringTok{documentclass[12pt,english]}\SpecialCharTok{\{article\}}
\CharTok{\textbackslash{}\textbackslash{}}\StringTok{usepackage[bottom]}\SpecialCharTok{\{footmisc\}}
\CharTok{\textbackslash{}\textbackslash{}}\StringTok{usepackage[urlcolor=blue]}\SpecialCharTok{\{hyperref\}}
\CharTok{\textbackslash{}\textbackslash{}}\StringTok{begin}\SpecialCharTok{\{document\}}
\CharTok{\textbackslash{}\textbackslash{}}\StringTok{title\{A Latex Testing File\}}
\CharTok{\textbackslash{}\textbackslash{}}\StringTok{author\{}\CharTok{\textbackslash{}\textbackslash{}}\StringTok{href\{http://fanwangecon.github.io/\}\{Fan Wang\} }\CharTok{\textbackslash{}\textbackslash{}}\StringTok{thanks\{See information }\CharTok{\textbackslash{}\textbackslash{}}\StringTok{href\{https://fanwangecon.github.io/Tex4Econ/\}}\SpecialCharTok{\{Tex4Econ\}}\StringTok{ for more.}\SpecialCharTok{\}\}}
\CharTok{\textbackslash{}\textbackslash{}}\StringTok{maketitle}
\StringTok{Ipsum information dolor sit amet, consectetur adipiscing elit. Integer Latex placerat nunc orci.}
\CharTok{\textbackslash{}\textbackslash{}}\StringTok{paragraph\{}\CharTok{\textbackslash{}\textbackslash{}}\StringTok{href\{https://papers.ssrn.com/sol3/papers.cfm?abstract_id=3140132\}}\SpecialCharTok{\{Data\}}\StringTok{\}}
\StringTok{Village closure information is taken from a village head survey.}\CharTok{\textbackslash{}\textbackslash{}}\StringTok{footnote\{Generally students went to schools.\}}
\CharTok{\textbackslash{}\textbackslash{}}\StringTok{end}\SpecialCharTok{\{document\}}\StringTok{"""}
\CommentTok{# Print}
\BuiltInTok{print}\NormalTok{(stf_tex_contents)}
\end{Highlighting}
\end{Shaded}

\begin{verbatim}
## \documentclass[12pt,english]{article}
## \usepackage[bottom]{footmisc}
## \usepackage[urlcolor=blue]{hyperref}
## \begin{document}
## \title{A Latex Testing File}
## \author{\href{http://fanwangecon.github.io/}{Fan Wang} \thanks{See information \href{https://fanwangecon.github.io/Tex4Econ/}{Tex4Econ} for more.}}
## \maketitle
## Ipsum information dolor sit amet, consectetur adipiscing elit. Integer Latex placerat nunc orci.
## \paragraph{\href{https://papers.ssrn.com/sol3/papers.cfm?abstract_id=3140132}{Data}}
## Village closure information is taken from a village head survey.\footnote{Generally students went to schools.}
## \end{document}
\end{verbatim}

Second, write the contents of the file to a new tex file stored inside the *\_file* subfolder of the directory:

\begin{Shaded}
\begin{Highlighting}[]
\CommentTok{# Relative file name}
\NormalTok{srt_file_tex }\OperatorTok{=} \StringTok{"_file/"}
\NormalTok{sna_file_tex }\OperatorTok{=} \StringTok{"test_fan"}
\NormalTok{srn_file_tex }\OperatorTok{=}\NormalTok{ srt_file_tex }\OperatorTok{+}\NormalTok{ sna_file_tex }\OperatorTok{+} \StringTok{".tex"}
\CommentTok{# Open new file}
\NormalTok{fl_tex_contents }\OperatorTok{=} \BuiltInTok{open}\NormalTok{(srn_file_tex, }\StringTok{'w'}\NormalTok{)}
\CommentTok{# Write to File}
\NormalTok{fl_tex_contents.write(stf_tex_contents)}
\CommentTok{# print}
\end{Highlighting}
\end{Shaded}

\begin{verbatim}
## 617
\end{verbatim}

\begin{Shaded}
\begin{Highlighting}[]
\NormalTok{fl_tex_contents.close()}
\end{Highlighting}
\end{Shaded}

\hypertarget{replace-strings-in-a-tex-file}{%
\subsubsection{Replace Strings in a tex file}\label{replace-strings-in-a-tex-file}}

Replace a set of strings in the file just generated by a set of alternative strings.

\begin{Shaded}
\begin{Highlighting}[]
\CommentTok{# Open file Get text}
\NormalTok{fl_tex_contents }\OperatorTok{=} \BuiltInTok{open}\NormalTok{(srn_file_tex)}
\NormalTok{stf_tex_contents }\OperatorTok{=}\NormalTok{ fl_tex_contents.read()}
\BuiltInTok{print}\NormalTok{(srn_file_tex)}

\CommentTok{# define new and old}
\end{Highlighting}
\end{Shaded}

\begin{verbatim}
## _file/test_fan.tex
\end{verbatim}

\begin{Shaded}
\begin{Highlighting}[]
\NormalTok{ls_st_old }\OperatorTok{=}\NormalTok{ [}\StringTok{'information'}\NormalTok{, }\StringTok{'Latex'}\NormalTok{]}
\NormalTok{ls_st_new }\OperatorTok{=}\NormalTok{ [}\StringTok{'INFOREPLACE'}\NormalTok{, }\StringTok{'LATEX'}\NormalTok{]}

\CommentTok{# zip and loop and replace}
\ControlFlowTok{for}\NormalTok{ old, new }\KeywordTok{in} \BuiltInTok{zip}\NormalTok{(ls_st_old, ls_st_new):}
\NormalTok{  stf_tex_contents }\OperatorTok{=}\NormalTok{ stf_tex_contents.replace(old, new)}
\BuiltInTok{print}\NormalTok{(stf_tex_contents)}

\CommentTok{# write to file with replacements}
\end{Highlighting}
\end{Shaded}

\begin{verbatim}
## \documentclass[12pt,english]{article}
## \usepackage[bottom]{footmisc}
## \usepackage[urlcolor=blue]{hyperref}
## \begin{document}
## \title{A LATEX Testing File}
## \author{\href{http://fanwangecon.github.io/}{Fan Wang} \thanks{See INFOREPLACE \href{https://fanwangecon.github.io/Tex4Econ/}{Tex4Econ} for more.}}
## \maketitle
## Ipsum INFOREPLACE dolor sit amet, consectetur adipiscing elit. Integer LATEX placerat nunc orci.
## \paragraph{\href{https://papers.ssrn.com/sol3/papers.cfm?abstract_id=3140132}{Data}}
## Village closure INFOREPLACE is taken from a village head survey.\footnote{Generally students went to schools.}
## \end{document}
\end{verbatim}

\begin{Shaded}
\begin{Highlighting}[]
\NormalTok{sna_file_edited_tex }\OperatorTok{=} \StringTok{"test_fan_edited"}
\NormalTok{srn_file_edited_tex }\OperatorTok{=}\NormalTok{ srt_file_tex }\OperatorTok{+}\NormalTok{ sna_file_edited_tex }\OperatorTok{+} \StringTok{".tex"}
\NormalTok{fl_tex_ed_contents }\OperatorTok{=} \BuiltInTok{open}\NormalTok{(srn_file_edited_tex, }\StringTok{'w'}\NormalTok{)}
\NormalTok{fl_tex_ed_contents.write(stf_tex_contents)}
\end{Highlighting}
\end{Shaded}

\begin{verbatim}
## 617
\end{verbatim}

\begin{Shaded}
\begin{Highlighting}[]
\NormalTok{fl_tex_ed_contents.close()}
\end{Highlighting}
\end{Shaded}

\hypertarget{convert-tex-file-to-pandoc-and-compile-latex}{%
\subsubsection{Convert Tex File to Pandoc and Compile Latex}\label{convert-tex-file-to-pandoc-and-compile-latex}}

Compile tex file to pdf and clean up the extraneous pdf outputs. See \href{https://pyfan.readthedocs.io/en/latest/autoapi/pyfan/util/pdf/pdfgen/index.html\#pyfan.util.pdf.pdfgen.ff_pdf_gen_clean}{ff\_pdf\_gen\_clean}.

\begin{Shaded}
\begin{Highlighting}[]
\ImportTok{import}\NormalTok{ subprocess}
\ImportTok{import}\NormalTok{ os}

\CommentTok{# Change to local directory so path in tex respected.}
\NormalTok{os.chdir(}\StringTok{"C:/Users/fan/pyfan/vig/support/inout"}\NormalTok{)}

\CommentTok{# Convert tex to pdf}
\NormalTok{subprocess.call([}\StringTok{'C:/texlive/2019/bin/win32/xelatex.exe'}\NormalTok{, }\StringTok{'-output-directory'}\NormalTok{,}
\NormalTok{                 srt_file_tex, srn_file_edited_tex], shell}\OperatorTok{=}\VariableTok{False}\NormalTok{)}
\CommentTok{# Clean pdf extraneous output}
\end{Highlighting}
\end{Shaded}

\begin{verbatim}
## 0
\end{verbatim}

\begin{Shaded}
\begin{Highlighting}[]
\NormalTok{ls_st_remove_suffix }\OperatorTok{=}\NormalTok{ [}\StringTok{'aux'}\NormalTok{,}\StringTok{'log'}\NormalTok{,}\StringTok{'out'}\NormalTok{,}\StringTok{'bbl'}\NormalTok{,}\StringTok{'blg'}\NormalTok{]}
\ControlFlowTok{for}\NormalTok{ st_suffix }\KeywordTok{in}\NormalTok{ ls_st_remove_suffix:}
\NormalTok{    srn_cur_file }\OperatorTok{=}\NormalTok{ srt_file_tex }\OperatorTok{+}\NormalTok{ sna_file_edited_tex }\OperatorTok{+} \StringTok{"."} \OperatorTok{+}\NormalTok{ st_suffix}
    \ControlFlowTok{if}\NormalTok{ (os.path.isfile(srn_cur_file)):}
\NormalTok{        os.remove(srt_file_tex }\OperatorTok{+}\NormalTok{ sna_file_edited_tex }\OperatorTok{+} \StringTok{"."} \OperatorTok{+}\NormalTok{ st_suffix)}
\end{Highlighting}
\end{Shaded}

Use pandoc to convert tex file

\begin{Shaded}
\begin{Highlighting}[]
\ImportTok{import}\NormalTok{ subprocess}

\CommentTok{# md file name}
\NormalTok{srn_file_md }\OperatorTok{=}\NormalTok{ srt_file_tex }\OperatorTok{+} \StringTok{"test_fan_edited.md"}
\CommentTok{# Convert tex to md}
\NormalTok{subprocess.call([}\StringTok{'pandoc'}\NormalTok{, }\StringTok{'-s'}\NormalTok{, srn_file_tex, }\StringTok{'-o'}\NormalTok{, srn_file_md])}
\CommentTok{# Open md file}
\end{Highlighting}
\end{Shaded}

\begin{verbatim}
## 0
\end{verbatim}

\begin{Shaded}
\begin{Highlighting}[]
\NormalTok{fl_md_contents }\OperatorTok{=} \BuiltInTok{open}\NormalTok{(srn_file_md)}
\BuiltInTok{print}\NormalTok{(fl_md_contents.read())}
\end{Highlighting}
\end{Shaded}

\begin{verbatim}
## ---
## author:
## - '[Fan Wang](http://fanwangecon.github.io/) [^1]'
## title: A Latex Testing File
## ---
## 
## \maketitle
## Ipsum information dolor sit amet, consectetur adipiscing elit. Integer
## Latex placerat nunc orci.
## 
## #### [Data](https://papers.ssrn.com/sol3/papers.cfm?abstract_id=3140132)
## 
## Village closure information is taken from a village head survey.[^2]
## 
## [^1]: See information
##     [Tex4Econ](https://fanwangecon.github.io/Tex4Econ/) for more.
## 
## [^2]: Generally students went to schools.
\end{verbatim}

\hypertarget{search-for-files-with-suffix-in-several-folders}{%
\subsubsection{Search for Files with Suffix in Several Folders}\label{search-for-files-with-suffix-in-several-folders}}

\begin{itemize}
\tightlist
\item
  python search all files in folders with suffix
\end{itemize}

Search for files in several directories that have a particular suffix. Then decompose directory into sub-components.

Search file inside several folders (not recursively in subfolders):

\begin{Shaded}
\begin{Highlighting}[]
\ImportTok{from}\NormalTok{ pathlib }\ImportTok{import}\NormalTok{ Path}

\CommentTok{# directories to search in}
\NormalTok{ls_spt_srh }\OperatorTok{=}\NormalTok{ [}\StringTok{"C:/Users/fan/R4Econ/amto/"}\NormalTok{,}
              \StringTok{"C:/Users/fan/R4Econ/function/"}\NormalTok{]}

\CommentTok{# get file names in folders (not recursively)}
\NormalTok{ls_spn_found }\OperatorTok{=}\NormalTok{ [spn_file }\ControlFlowTok{for}\NormalTok{ spt_srh }\KeywordTok{in}\NormalTok{ ls_spt_srh}
                         \ControlFlowTok{for}\NormalTok{ spn_file }\KeywordTok{in}\NormalTok{ Path(spt_srh).glob(}\StringTok{'*.Rmd'}\NormalTok{)]}
\ControlFlowTok{for}\NormalTok{ spn_found }\KeywordTok{in}\NormalTok{ ls_spn_found:}
  \BuiltInTok{print}\NormalTok{(spn_found)}
\end{Highlighting}
\end{Shaded}

\begin{verbatim}
## C:\Users\fan\R4Econ\amto\main.Rmd
## C:\Users\fan\R4Econ\function\main.Rmd
\end{verbatim}

Search file recursivesly in all subfolders of folders:

\begin{Shaded}
\begin{Highlighting}[]
\ImportTok{from}\NormalTok{ pathlib }\ImportTok{import}\NormalTok{ Path}

\CommentTok{# directories to search in}
\NormalTok{ls_spt_srh }\OperatorTok{=}\NormalTok{ [}\StringTok{"C:/Users/fan/R4Econ/amto/array/"}\NormalTok{,}
              \StringTok{"C:/Users/fan/R4Econ/amto/list"}\NormalTok{]}

\CommentTok{# get file names recursively in all subfolders}
\NormalTok{ls_spn_found }\OperatorTok{=}\NormalTok{ [spn_file }\ControlFlowTok{for}\NormalTok{ spt_srh }\KeywordTok{in}\NormalTok{ ls_spt_srh}
                         \ControlFlowTok{for}\NormalTok{ spn_file }\KeywordTok{in}\NormalTok{ Path(spt_srh).rglob(}\StringTok{'*.R'}\NormalTok{)]}
\ControlFlowTok{for}\NormalTok{ spn_found }\KeywordTok{in}\NormalTok{ ls_spn_found:}
\NormalTok{  drive, path_and_file }\OperatorTok{=}\NormalTok{ os.path.splitdrive(spn_found)}
\NormalTok{  path_no_suffix }\OperatorTok{=}\NormalTok{ os.path.splitext(spn_found)[}\DecValTok{0}\NormalTok{]}
\NormalTok{  path_no_file, }\BuiltInTok{file} \OperatorTok{=}\NormalTok{ os.path.split(spn_found)}
\NormalTok{  file_no_suffix }\OperatorTok{=}\NormalTok{ Path(spn_found).stem}
  \BuiltInTok{print}\NormalTok{(}\StringTok{'file:'}\NormalTok{, }\BuiltInTok{file}\NormalTok{, }\StringTok{'}\CharTok{\textbackslash{}n}\StringTok{drive:'}\NormalTok{, drive,}
        \StringTok{'}\CharTok{\textbackslash{}n}\StringTok{file no suffix:'}\NormalTok{, file_no_suffix,}
        \StringTok{'}\CharTok{\textbackslash{}n}\StringTok{full path:'}\NormalTok{, spn_found,}
        \StringTok{'}\CharTok{\textbackslash{}n}\StringTok{pt no fle:'}\NormalTok{, path_no_file,}
        \StringTok{'}\CharTok{\textbackslash{}n}\StringTok{pt no suf:'}\NormalTok{, path_no_suffix, }\StringTok{'}\CharTok{\textbackslash{}n}\StringTok{'}\NormalTok{)}
\end{Highlighting}
\end{Shaded}

\begin{verbatim}
## file: fs_ary_basics.R 
## drive: C: 
## file no suffix: fs_ary_basics 
## full path: C:\Users\fan\R4Econ\amto\array\htmlpdfr\fs_ary_basics.R 
## pt no fle: C:\Users\fan\R4Econ\amto\array\htmlpdfr 
## pt no suf: C:\Users\fan\R4Econ\amto\array\htmlpdfr\fs_ary_basics 
## 
## file: fs_ary_generate.R 
## drive: C: 
## file no suffix: fs_ary_generate 
## full path: C:\Users\fan\R4Econ\amto\array\htmlpdfr\fs_ary_generate.R 
## pt no fle: C:\Users\fan\R4Econ\amto\array\htmlpdfr 
## pt no suf: C:\Users\fan\R4Econ\amto\array\htmlpdfr\fs_ary_generate 
## 
## file: fs_ary_mesh.R 
## drive: C: 
## file no suffix: fs_ary_mesh 
## full path: C:\Users\fan\R4Econ\amto\array\htmlpdfr\fs_ary_mesh.R 
## pt no fle: C:\Users\fan\R4Econ\amto\array\htmlpdfr 
## pt no suf: C:\Users\fan\R4Econ\amto\array\htmlpdfr\fs_ary_mesh 
## 
## file: fs_ary_string.R 
## drive: C: 
## file no suffix: fs_ary_string 
## full path: C:\Users\fan\R4Econ\amto\array\htmlpdfr\fs_ary_string.R 
## pt no fle: C:\Users\fan\R4Econ\amto\array\htmlpdfr 
## pt no suf: C:\Users\fan\R4Econ\amto\array\htmlpdfr\fs_ary_string 
## 
## file: fs_listr.R 
## drive: C: 
## file no suffix: fs_listr 
## full path: C:\Users\fan\R4Econ\amto\list\htmlpdfr\fs_listr.R 
## pt no fle: C:\Users\fan\R4Econ\amto\list\htmlpdfr 
## pt no suf: C:\Users\fan\R4Econ\amto\list\htmlpdfr\fs_listr 
## 
## file: fs_lst_basics.R 
## drive: C: 
## file no suffix: fs_lst_basics 
## full path: C:\Users\fan\R4Econ\amto\list\htmlpdfr\fs_lst_basics.R 
## pt no fle: C:\Users\fan\R4Econ\amto\list\htmlpdfr 
## pt no suf: C:\Users\fan\R4Econ\amto\list\htmlpdfr\fs_lst_basics
\end{verbatim}

\hypertarget{folder-operations}{%
\subsection{Folder Operations}\label{folder-operations}}

\begin{quote}
Go to the \href{https://github.com/FanWangEcon//pyfan/blob/master/.//A-Collection-of-Python-Examples.Rmd}{\textbf{RMD}}, \href{https://github.com/FanWangEcon//pyfan/blob/master/.//htmlpdfr/A-Collection-of-Python-Examples.pdf}{\textbf{PDF}}, or \href{https://fanwangecon.github.io//pyfan/.//htmlpdfr/A-Collection-of-Python-Examples.html}{\textbf{HTML}} version of this file. Go back to \href{http://fanwangecon.github.io/}{fan}'s \href{https://fanwangecon.github.io/pyfan/}{Python Code Examples} Repository (\href{https://fanwangecon.github.io/pyfan/bookdown}{bookdown site}).
\end{quote}

\hypertarget{new-folder-and-files}{%
\subsubsection{New Folder and Files}\label{new-folder-and-files}}

\begin{enumerate}
\def\labelenumi{\arabic{enumi}.}
\tightlist
\item
  create a folder and subfolder
\item
  create two files in the new folder
\end{enumerate}

\begin{Shaded}
\begin{Highlighting}[]
\ImportTok{import}\NormalTok{ pathlib}

\CommentTok{# folder root}
\NormalTok{srt_folder }\OperatorTok{=} \StringTok{"_folder/"}

\CommentTok{# new folder}
\NormalTok{srt_subfolder }\OperatorTok{=}\NormalTok{ srt_folder }\OperatorTok{+} \StringTok{"fa/"}
\CommentTok{# new subfolder}
\NormalTok{srt_subfolder }\OperatorTok{=}\NormalTok{ srt_subfolder }\OperatorTok{+} \StringTok{"faa/"}
\CommentTok{# generate folders recursively}
\NormalTok{pathlib.Path(srt_subfolder).mkdir(parents}\OperatorTok{=}\VariableTok{True}\NormalTok{, exist_ok}\OperatorTok{=}\VariableTok{True}\NormalTok{)}

\CommentTok{# Open new file}
\NormalTok{fl_tex_contents_aa }\OperatorTok{=} \BuiltInTok{open}\NormalTok{(srt_subfolder }\OperatorTok{+} \StringTok{"file_a.txt"}\NormalTok{, }\StringTok{'w'}\NormalTok{)}
\CommentTok{# Write to File}
\NormalTok{fl_tex_contents_aa.write(}\StringTok{'contents of file a'}\NormalTok{)}
\end{Highlighting}
\end{Shaded}

\begin{verbatim}
## 18
\end{verbatim}

\begin{Shaded}
\begin{Highlighting}[]
\NormalTok{fl_tex_contents_aa.close()}

\CommentTok{# Open another new file and save}
\NormalTok{fl_tex_contents_ab }\OperatorTok{=} \BuiltInTok{open}\NormalTok{(srt_subfolder }\OperatorTok{+} \StringTok{"file_b.txt"}\NormalTok{, }\StringTok{'w'}\NormalTok{)}
\CommentTok{# Write to File}
\NormalTok{fl_tex_contents_ab.write(}\StringTok{'contents of file b'}\NormalTok{)}
\end{Highlighting}
\end{Shaded}

\begin{verbatim}
## 18
\end{verbatim}

\begin{Shaded}
\begin{Highlighting}[]
\NormalTok{fl_tex_contents_ab.close()}
\end{Highlighting}
\end{Shaded}

Generate more folders without files:

\begin{Shaded}
\begin{Highlighting}[]
\CommentTok{# generate folders recursively}
\NormalTok{pathlib.Path(}\StringTok{"_folder/fb/fba/"}\NormalTok{).mkdir(parents}\OperatorTok{=}\VariableTok{True}\NormalTok{, exist_ok}\OperatorTok{=}\VariableTok{True}\NormalTok{)}
\CommentTok{# generate folders recursively}
\NormalTok{pathlib.Path(}\StringTok{"_folder/fc/"}\NormalTok{).mkdir(parents}\OperatorTok{=}\VariableTok{True}\NormalTok{, exist_ok}\OperatorTok{=}\VariableTok{True}\NormalTok{)}
\CommentTok{# generate folders recursively}
\NormalTok{pathlib.Path(}\StringTok{"_folder/fd/"}\NormalTok{).mkdir(parents}\OperatorTok{=}\VariableTok{True}\NormalTok{, exist_ok}\OperatorTok{=}\VariableTok{True}\NormalTok{)}
\end{Highlighting}
\end{Shaded}

\hypertarget{copy-a-file-from-one-folder-to-another}{%
\subsubsection{Copy a File from One Folder to Another}\label{copy-a-file-from-one-folder-to-another}}

Move the two files from *\_folder/fa/faa* to *\_folder/faa* as well as to *\_folder/fb/faa\emph{. Use }shutil.copy2* so that more metadata is copied over. But \emph{copyfile} is faster.

\begin{itemize}
\tightlist
\item
  \href{https://stackoverflow.com/a/123238/8280804}{How do I copy a file in Python?}
\end{itemize}

Moving one file:

\begin{Shaded}
\begin{Highlighting}[]
\ImportTok{import}\NormalTok{ shutil}
\CommentTok{# Faster method}
\NormalTok{shutil.copyfile(}\StringTok{'_folder/fa/faa/file_a.txt'}\NormalTok{, }\StringTok{'_folder/fb/file_a.txt'}\NormalTok{)}
\CommentTok{# More metadat copied, and don't need to specify name }
\end{Highlighting}
\end{Shaded}

\begin{verbatim}
## '_folder/fb/file_a.txt'
\end{verbatim}

\begin{Shaded}
\begin{Highlighting}[]
\NormalTok{shutil.copy2(}\StringTok{'_folder/fa/faa/file_a.txt'}\NormalTok{, }\StringTok{'_folder/fb/fba'}\NormalTok{)}
\end{Highlighting}
\end{Shaded}

\begin{verbatim}
## '_folder/fb/fba\\file_a.txt'
\end{verbatim}

\hypertarget{copy-folder-to-multiple-destimations}{%
\subsubsection{Copy Folder to Multiple Destimations}\label{copy-folder-to-multiple-destimations}}

Move Entire Folder, \href{https://stackoverflow.com/a/31039095/8280804}{How do I copy an entire directory of files into an existing directory using Python?}:

\begin{Shaded}
\begin{Highlighting}[]
\ImportTok{from}\NormalTok{ distutils.dir_util }\ImportTok{import}\NormalTok{ copy_tree}

\CommentTok{# Move contents from fa/faa/ to fc/faa}
\NormalTok{srt_curroot }\OperatorTok{=} \StringTok{'_folder/fa/'}
\NormalTok{srt_folder }\OperatorTok{=} \StringTok{'faa/'}
\NormalTok{srt_newroot }\OperatorTok{=} \StringTok{'_folder/fc/'}

\CommentTok{# Full source and destination}
\NormalTok{srt_sourc }\OperatorTok{=}\NormalTok{ srt_curroot }\OperatorTok{+}\NormalTok{ srt_folder}
\NormalTok{srt_desct }\OperatorTok{=}\NormalTok{ srt_newroot }\OperatorTok{+}\NormalTok{ srt_folder}

\CommentTok{# Check/Create new Directory }
\NormalTok{pathlib.Path(srt_desct).mkdir(parents}\OperatorTok{=}\VariableTok{True}\NormalTok{, exist_ok}\OperatorTok{=}\VariableTok{True}\NormalTok{)}

\CommentTok{# Move}
\NormalTok{copy_tree(srt_sourc, srt_desct)}
\end{Highlighting}
\end{Shaded}

\begin{verbatim}
## ['_folder/fc/faa/file_a.txt', '_folder/fc/faa/file_b.txt']
\end{verbatim}

Move contents to multiple destinations:

\begin{Shaded}
\begin{Highlighting}[]
\ImportTok{from}\NormalTok{ distutils.dir_util }\ImportTok{import}\NormalTok{ copy_tree}
\CommentTok{# Check/Create new Directory }
\NormalTok{pathlib.Path(}\StringTok{'_folder/fd/faa/fa_images'}\NormalTok{).mkdir(parents}\OperatorTok{=}\VariableTok{True}\NormalTok{, exist_ok}\OperatorTok{=}\VariableTok{True}\NormalTok{)}
\NormalTok{pathlib.Path(}\StringTok{'_folder/fd/faa/fb_images'}\NormalTok{).mkdir(parents}\OperatorTok{=}\VariableTok{True}\NormalTok{, exist_ok}\OperatorTok{=}\VariableTok{True}\NormalTok{)}
\NormalTok{pathlib.Path(}\StringTok{'_folder/fd/faa/fc_images'}\NormalTok{).mkdir(parents}\OperatorTok{=}\VariableTok{True}\NormalTok{, exist_ok}\OperatorTok{=}\VariableTok{True}\NormalTok{)}
\NormalTok{pathlib.Path(}\StringTok{'_folder/fd/faa/fz_img'}\NormalTok{).mkdir(parents}\OperatorTok{=}\VariableTok{True}\NormalTok{, exist_ok}\OperatorTok{=}\VariableTok{True}\NormalTok{)}
\NormalTok{pathlib.Path(}\StringTok{'_folder/fd/faa/fz_other'}\NormalTok{).mkdir(parents}\OperatorTok{=}\VariableTok{True}\NormalTok{, exist_ok}\OperatorTok{=}\VariableTok{True}\NormalTok{)}

\CommentTok{# Move}
\NormalTok{copy_tree(}\StringTok{'_folder/fa/faa/'}\NormalTok{, }\StringTok{'_folder/fd/faa/fa_images'}\NormalTok{)}
\end{Highlighting}
\end{Shaded}

\begin{verbatim}
## ['_folder/fd/faa/fa_images\\file_a.txt', '_folder/fd/faa/fa_images\\file_b.txt']
\end{verbatim}

\begin{Shaded}
\begin{Highlighting}[]
\NormalTok{copy_tree(}\StringTok{'_folder/fa/faa/'}\NormalTok{, }\StringTok{'_folder/fd/faa/fb_images'}\NormalTok{)}
\end{Highlighting}
\end{Shaded}

\begin{verbatim}
## ['_folder/fd/faa/fb_images\\file_a.txt', '_folder/fd/faa/fb_images\\file_b.txt']
\end{verbatim}

\begin{Shaded}
\begin{Highlighting}[]
\NormalTok{copy_tree(}\StringTok{'_folder/fa/faa/'}\NormalTok{, }\StringTok{'_folder/fd/faa/fc_images'}\NormalTok{)}
\end{Highlighting}
\end{Shaded}

\begin{verbatim}
## ['_folder/fd/faa/fc_images\\file_a.txt', '_folder/fd/faa/fc_images\\file_b.txt']
\end{verbatim}

\begin{Shaded}
\begin{Highlighting}[]
\NormalTok{copy_tree(}\StringTok{'_folder/fa/faa/'}\NormalTok{, }\StringTok{'_folder/fd/faa/fz_img'}\NormalTok{)}
\end{Highlighting}
\end{Shaded}

\begin{verbatim}
## ['_folder/fd/faa/fz_img\\file_a.txt', '_folder/fd/faa/fz_img\\file_b.txt']
\end{verbatim}

\begin{Shaded}
\begin{Highlighting}[]
\NormalTok{copy_tree(}\StringTok{'_folder/fa/faa/'}\NormalTok{, }\StringTok{'_folder/fd/faa/fz_other'}\NormalTok{)}
\CommentTok{# Empty Folder}
\end{Highlighting}
\end{Shaded}

\begin{verbatim}
## ['_folder/fd/faa/fz_other\\file_a.txt', '_folder/fd/faa/fz_other\\file_b.txt']
\end{verbatim}

\begin{Shaded}
\begin{Highlighting}[]
\NormalTok{pathlib.Path(}\StringTok{'_folder/fd/faa/fd_images'}\NormalTok{).mkdir(parents}\OperatorTok{=}\VariableTok{True}\NormalTok{, exist_ok}\OperatorTok{=}\VariableTok{True}\NormalTok{)}
\NormalTok{pathlib.Path(}\StringTok{'_folder/fd/faa/fe_images'}\NormalTok{).mkdir(parents}\OperatorTok{=}\VariableTok{True}\NormalTok{, exist_ok}\OperatorTok{=}\VariableTok{True}\NormalTok{)}
\end{Highlighting}
\end{Shaded}

\hypertarget{search-for-files-in-folder}{%
\subsubsection{Search for Files in Folder}\label{search-for-files-in-folder}}

Find the total number of files in a folder.

\begin{Shaded}
\begin{Highlighting}[]
\ImportTok{import}\NormalTok{ pathlib}
\CommentTok{# the number of files in folder found with search critiera}
\NormalTok{st_fle_search }\OperatorTok{=} \StringTok{'*.txt'}
\NormalTok{ls_spn }\OperatorTok{=}\NormalTok{ [Path(spn).stem }\ControlFlowTok{for}\NormalTok{ spn }\KeywordTok{in}\NormalTok{ Path(}\StringTok{'_folder/fd/faa/fa_images'}\NormalTok{).rglob(st_fle_search)]}
\BuiltInTok{print}\NormalTok{(ls_spn)}

\CommentTok{# count files in a non-empty folder}
\end{Highlighting}
\end{Shaded}

\begin{verbatim}
## ['file_a', 'file_b']
\end{verbatim}

\begin{Shaded}
\begin{Highlighting}[]
\NormalTok{srn }\OperatorTok{=} \StringTok{'_folder/fd/faa/fa_images'}
\NormalTok{[spn }\ControlFlowTok{for}\NormalTok{ spn }\KeywordTok{in}\NormalTok{ Path(srn).rglob(st_fle_search)]}
\end{Highlighting}
\end{Shaded}

\begin{verbatim}
## [WindowsPath('_folder/fd/faa/fa_images/file_a.txt'), WindowsPath('_folder/fd/faa/fa_images/file_b.txt')]
\end{verbatim}

\begin{Shaded}
\begin{Highlighting}[]
\NormalTok{bl_folder_is_empty }\OperatorTok{=} \BuiltInTok{len}\NormalTok{([spn }\ControlFlowTok{for}\NormalTok{ spn }\KeywordTok{in}\NormalTok{ Path(srn).rglob(st_fle_search)])}\OperatorTok{>}\DecValTok{0}
\BuiltInTok{print}\NormalTok{(bl_folder_is_empty)}

\CommentTok{# count files in an empty folder}
\end{Highlighting}
\end{Shaded}

\begin{verbatim}
## True
\end{verbatim}

\begin{Shaded}
\begin{Highlighting}[]
\NormalTok{srn }\OperatorTok{=} \StringTok{'_folder/fd/faa/fd_images'}
\NormalTok{[spn }\ControlFlowTok{for}\NormalTok{ spn }\KeywordTok{in}\NormalTok{ Path(srn).rglob(st_fle_search)]}
\end{Highlighting}
\end{Shaded}

\begin{verbatim}
## []
\end{verbatim}

\begin{Shaded}
\begin{Highlighting}[]
\NormalTok{bl_folder_is_empty }\OperatorTok{=} \BuiltInTok{len}\NormalTok{([spn }\ControlFlowTok{for}\NormalTok{ spn }\KeywordTok{in}\NormalTok{ Path(srn).rglob(st_fle_search)])}\OperatorTok{>}\DecValTok{0}
\BuiltInTok{print}\NormalTok{(bl_folder_is_empty)}
\end{Highlighting}
\end{Shaded}

\begin{verbatim}
## False
\end{verbatim}

\hypertarget{search-for-folder-names}{%
\subsubsection{Search for Folder Names}\label{search-for-folder-names}}

\begin{itemize}
\tightlist
\item
  \href{https://stackoverflow.com/a/40404282/8280804}{python search for folders containing strings}
\end{itemize}

Search for folders with certain search word in folder name, and only keep if folder actually has files.

\begin{Shaded}
\begin{Highlighting}[]
\ImportTok{import}\NormalTok{ os}

\CommentTok{# get all folder names in folder}
\NormalTok{ls_spt }\OperatorTok{=}\NormalTok{ os.listdir(}\StringTok{'_folder/fd/faa/'}\NormalTok{)}
\BuiltInTok{print}\NormalTok{(ls_spt)}

\CommentTok{# Select only subfolder names containing _images}
\end{Highlighting}
\end{Shaded}

\begin{verbatim}
## ['fa_images', 'fb_images', 'fc_images', 'fd_images', 'fe_images', 'fz_img', 'fz_other', '_img']
\end{verbatim}

\begin{Shaded}
\begin{Highlighting}[]
\NormalTok{srt }\OperatorTok{=} \StringTok{'_folder/fd/faa/'}
\NormalTok{st_search }\OperatorTok{=} \StringTok{'_images'}
\NormalTok{ls_srt_found }\OperatorTok{=}\NormalTok{ [srt }\OperatorTok{+}\NormalTok{ spt }
                \ControlFlowTok{for}\NormalTok{ spt }\KeywordTok{in}\NormalTok{ os.listdir(srt) }
                \ControlFlowTok{if}\NormalTok{ st_search }\KeywordTok{in}\NormalTok{ spt]}
\BuiltInTok{print}\NormalTok{(ls_srt_found)}
\end{Highlighting}
\end{Shaded}

\begin{verbatim}
## ['_folder/fd/faa/fa_images', '_folder/fd/faa/fb_images', '_folder/fd/faa/fc_images', '_folder/fd/faa/fd_images', '_folder/fd/faa/fe_images']
\end{verbatim}

\hypertarget{find-non-empty-folders-by-name}{%
\subsubsection{Find Non-empty Folders by Name}\label{find-non-empty-folders-by-name}}

Search:

\begin{enumerate}
\def\labelenumi{\arabic{enumi}.}
\tightlist
\item
  Get subfolders in folder with string in name
\item
  Only collect if there are files in the subfolder
\end{enumerate}

\begin{Shaded}
\begin{Highlighting}[]
\ImportTok{import}\NormalTok{ pathlib}

\CommentTok{# Select only subfolder names containing _images}
\NormalTok{srt }\OperatorTok{=} \StringTok{'_folder/fd/faa/'}
\CommentTok{# the folder names must contain _images}
\NormalTok{st_srt_srh }\OperatorTok{=} \StringTok{'_images'}
\CommentTok{# there must be files in the folder with this string}
\NormalTok{st_fle_srh }\OperatorTok{=} \StringTok{'*.txt'}

\CommentTok{# All folders that have String}
\NormalTok{ls_srt_found }\OperatorTok{=}\NormalTok{ [srt }\OperatorTok{+}\NormalTok{ spt }
                \ControlFlowTok{for}\NormalTok{ spt }\KeywordTok{in}\NormalTok{ os.listdir(srt) }
                \ControlFlowTok{if}\NormalTok{ st_srt_srh }\KeywordTok{in}\NormalTok{ spt]}
\BuiltInTok{print}\NormalTok{(ls_srt_found)}

\CommentTok{# All folders that have String and that are nonempty}
\end{Highlighting}
\end{Shaded}

\begin{verbatim}
## ['_folder/fd/faa/fa_images', '_folder/fd/faa/fb_images', '_folder/fd/faa/fc_images', '_folder/fd/faa/fd_images', '_folder/fd/faa/fe_images']
\end{verbatim}

\begin{Shaded}
\begin{Highlighting}[]
\NormalTok{ls_srt_found }\OperatorTok{=}\NormalTok{ [srt }\OperatorTok{+}\NormalTok{ spt}
                \ControlFlowTok{for}\NormalTok{ spt }\KeywordTok{in}\NormalTok{ os.listdir(srt)}
                \ControlFlowTok{if}\NormalTok{ ((st_srt_srh }\KeywordTok{in}\NormalTok{ spt) }
                    \KeywordTok{and} 
\NormalTok{                    (}\BuiltInTok{len}\NormalTok{([spn }\ControlFlowTok{for}\NormalTok{ spn }
                          \KeywordTok{in}\NormalTok{ Path(srt }\OperatorTok{+}\NormalTok{ spt).rglob(st_fle_srh)])}\OperatorTok{>}\DecValTok{0}\NormalTok{)) ]}
\BuiltInTok{print}\NormalTok{(ls_srt_found)}
\end{Highlighting}
\end{Shaded}

\begin{verbatim}
## ['_folder/fd/faa/fa_images', '_folder/fd/faa/fb_images', '_folder/fd/faa/fc_images']
\end{verbatim}

\hypertarget{found-folders-to-new-folder}{%
\subsubsection{Found Folders to new Folder}\label{found-folders-to-new-folder}}

\begin{enumerate}
\def\labelenumi{\arabic{enumi}.}
\tightlist
\item
  Search for subfolders by folder name string in a folder
\item
  Select nonempty subfolders
\item
  Move nonsempty subfolders to one new folder
\item
  Move this single combination folder
\end{enumerate}

The results here are implemented as function in the \href{https://github.com/FanWangEcon/pyfan}{pyfan} package: \href{https://pyfan.readthedocs.io/en/latest/autoapi/pyfan/util/path/movefiles/index.html\#pyfan.util.path.movefiles.fp_agg_move_subfiles}{fp\_agg\_move\_subfiles}.

\begin{Shaded}
\begin{Highlighting}[]
\ImportTok{import}\NormalTok{ pathlib}
\ImportTok{import}\NormalTok{ os}
\ImportTok{import}\NormalTok{ shutil}
\ImportTok{from}\NormalTok{ distutils.dir_util }\ImportTok{import}\NormalTok{ copy_tree}

\CommentTok{# 1 Define Parameters}

\CommentTok{# Select only subfolder names containing _images}
\NormalTok{srt }\OperatorTok{=} \StringTok{'_folder/fd/faa/'}
\CommentTok{# the folder names must contain _images}
\NormalTok{st_srt_srh }\OperatorTok{=} \StringTok{'_images'}
\CommentTok{# there must be files in the folder with this string}
\NormalTok{st_fle_srh }\OperatorTok{=} \StringTok{'*.txt'}

\CommentTok{# new aggregating folder name}
\NormalTok{srt_agg }\OperatorTok{=} \StringTok{'_img'}

\CommentTok{# folders to move aggregation files towards}
\NormalTok{ls_srt_dest }\OperatorTok{=}\NormalTok{ [}\StringTok{'_folder/fd/faa/'}\NormalTok{, }\StringTok{'_folder/'}\NormalTok{]}

\CommentTok{# delete source}
\NormalTok{bl_delete_source }\OperatorTok{=} \VariableTok{False}

\CommentTok{# 2 Gather Folders }
\NormalTok{ls_ls_srt_found }\OperatorTok{=}\NormalTok{ [[srt }\OperatorTok{+}\NormalTok{ spt, spt]}
                    \ControlFlowTok{for}\NormalTok{ spt }\KeywordTok{in}\NormalTok{ os.listdir(srt)}
                    \ControlFlowTok{if}\NormalTok{ ((st_srt_srh }\KeywordTok{in}\NormalTok{ spt) }
                        \KeywordTok{and} 
\NormalTok{                        (}\BuiltInTok{len}\NormalTok{([spn }\ControlFlowTok{for}\NormalTok{ spn }
                              \KeywordTok{in}\NormalTok{ Path(srt }\OperatorTok{+}\NormalTok{ spt).rglob(st_fle_srh)])}\OperatorTok{>}\DecValTok{0}\NormalTok{)) ]}
\BuiltInTok{print}\NormalTok{(ls_ls_srt_found)}

\CommentTok{# 3 Loop over destination folders, loop over source folders}
\end{Highlighting}
\end{Shaded}

\begin{verbatim}
## [['_folder/fd/faa/fa_images', 'fa_images'], ['_folder/fd/faa/fb_images', 'fb_images'], ['_folder/fd/faa/fc_images', 'fc_images']]
\end{verbatim}

\begin{Shaded}
\begin{Highlighting}[]
\ControlFlowTok{for}\NormalTok{ srt }\KeywordTok{in}\NormalTok{ ls_srt_dest:}

  \CommentTok{# Move each folder over}
  \ControlFlowTok{for}\NormalTok{ ls_srt_found }\KeywordTok{in}\NormalTok{ ls_ls_srt_found:}

    \CommentTok{# Paths}
\NormalTok{    srt_source }\OperatorTok{=}\NormalTok{ ls_srt_found[}\DecValTok{0}\NormalTok{]}
\NormalTok{    srt_dest }\OperatorTok{=}\NormalTok{ os.path.join(srt, srt_agg, ls_srt_found[}\DecValTok{1}\NormalTok{])}
    
    \CommentTok{# dest folders}
\NormalTok{    pathlib.Path(srt_dest).mkdir(parents}\OperatorTok{=}\VariableTok{True}\NormalTok{, exist_ok}\OperatorTok{=}\VariableTok{True}\NormalTok{)}
    
    \CommentTok{# move}
\NormalTok{    copy_tree(ls_srt_found[}\DecValTok{0}\NormalTok{], srt_dest)}

\CommentTok{# 4. Delete Sources}
\end{Highlighting}
\end{Shaded}

\begin{verbatim}
## ['_folder/fd/faa/_img\\fa_images\\file_a.txt', '_folder/fd/faa/_img\\fa_images\\file_b.txt']
## ['_folder/fd/faa/_img\\fb_images\\file_a.txt', '_folder/fd/faa/_img\\fb_images\\file_b.txt']
## ['_folder/fd/faa/_img\\fc_images\\file_a.txt', '_folder/fd/faa/_img\\fc_images\\file_b.txt']
## ['_folder/_img\\fa_images\\file_a.txt', '_folder/_img\\fa_images\\file_b.txt']
## ['_folder/_img\\fb_images\\file_a.txt', '_folder/_img\\fb_images\\file_b.txt']
## ['_folder/_img\\fc_images\\file_a.txt', '_folder/_img\\fc_images\\file_b.txt']
\end{verbatim}

\begin{Shaded}
\begin{Highlighting}[]
\ControlFlowTok{if}\NormalTok{ bl_delete_source:}
  \ControlFlowTok{for}\NormalTok{ ls_srt_found }\KeywordTok{in}\NormalTok{ ls_ls_srt_found:}
\NormalTok{    shutil.rmtree(ls_srt_found[}\DecValTok{0}\NormalTok{])}
\end{Highlighting}
\end{Shaded}

\hypertarget{parse-yaml}{%
\subsection{Parse Yaml}\label{parse-yaml}}

\begin{quote}
Go to the \href{https://github.com/FanWangEcon//pyfan/blob/master/.//A-Collection-of-Python-Examples.Rmd}{\textbf{RMD}}, \href{https://github.com/FanWangEcon//pyfan/blob/master/.//htmlpdfr/A-Collection-of-Python-Examples.pdf}{\textbf{PDF}}, or \href{https://fanwangecon.github.io//pyfan/.//htmlpdfr/A-Collection-of-Python-Examples.html}{\textbf{HTML}} version of this file. Go back to \href{http://fanwangecon.github.io/}{fan}'s \href{https://fanwangecon.github.io/pyfan/}{Python Code Examples} Repository (\href{https://fanwangecon.github.io/pyfan/bookdown}{bookdown site}).
\end{quote}

Use the \href{https://pypi.org/project/PyYAML/}{PyYAML} to parse yaml.

\hypertarget{write-and-create-a-simple-yaml-file}{%
\subsubsection{Write and Create a Simple YAML file}\label{write-and-create-a-simple-yaml-file}}

First, Yaml as a string variable:

\begin{Shaded}
\begin{Highlighting}[]
\CommentTok{# Create the Tex Text}
\CommentTok{# Note that trible quotes begin first and end last lines}
\NormalTok{stf_tex_contents }\OperatorTok{=} \StringTok{"""\textbackslash{}}
\StringTok{- file: matrix_matlab}
\StringTok{  title: "One Variable Graphs and Tables"}
\StringTok{  description: |}
\StringTok{    Frequency table, bar chart and histogram.}
\StringTok{    R function and lapply to generate graphs/tables for different variables.}
\StringTok{  core:}
\StringTok{  - package: r}
\StringTok{    code: |}
\StringTok{      c('word1','word2')}
\StringTok{      function()}
\StringTok{      for (ctr in c(1,2)) }\SpecialCharTok{\{\}}
\StringTok{  - package: dplyr}
\StringTok{    code: |}
\StringTok{      group_by()}
\StringTok{  date: 2020-05-02}
\StringTok{  output:}
\StringTok{    pdf_document:}
\StringTok{      pandoc_args: '../_output_kniti_pdf.yaml'}
\StringTok{      includes:}
\StringTok{        in_header: '../preamble.tex'}
\StringTok{  urlcolor: blue}
\StringTok{- file: matrix_algebra_rules}
\StringTok{  title: "Opening a Dataset"}
\StringTok{  titleshort: "Opening a Dataset"}
\StringTok{  description: |}
\StringTok{    Opening a Dataset.}
\StringTok{  core:}
\StringTok{  - package: r}
\StringTok{    code: |}
\StringTok{      setwd()}
\StringTok{  - package: readr}
\StringTok{    code: |}
\StringTok{      write_csv()}
\StringTok{  date: 2020-05-02}
\StringTok{  date_start: 2018-12-01}
\StringTok{- file: matrix_two}
\StringTok{  title: "Third file"}
\StringTok{  titleshort: "Third file"}
\StringTok{  description: |}
\StringTok{    Third file description."""}
\CommentTok{# Print}
\BuiltInTok{print}\NormalTok{(stf_tex_contents)}
\end{Highlighting}
\end{Shaded}

\begin{verbatim}
## - file: matrix_matlab
##   title: "One Variable Graphs and Tables"
##   description: |
##     Frequency table, bar chart and histogram.
##     R function and lapply to generate graphs/tables for different variables.
##   core:
##   - package: r
##     code: |
##       c('word1','word2')
##       function()
##       for (ctr in c(1,2)) {}
##   - package: dplyr
##     code: |
##       group_by()
##   date: 2020-05-02
##   output:
##     pdf_document:
##       pandoc_args: '../_output_kniti_pdf.yaml'
##       includes:
##         in_header: '../preamble.tex'
##   urlcolor: blue
## - file: matrix_algebra_rules
##   title: "Opening a Dataset"
##   titleshort: "Opening a Dataset"
##   description: |
##     Opening a Dataset.
##   core:
##   - package: r
##     code: |
##       setwd()
##   - package: readr
##     code: |
##       write_csv()
##   date: 2020-05-02
##   date_start: 2018-12-01
## - file: matrix_two
##   title: "Third file"
##   titleshort: "Third file"
##   description: |
##     Third file description.
\end{verbatim}

Second, write the contents of the file to a new tex file stored inside the *\_file* subfolder of the directory:

\begin{Shaded}
\begin{Highlighting}[]
\CommentTok{# Relative file name}
\NormalTok{srt_file_tex }\OperatorTok{=} \StringTok{"_file/"}
\NormalTok{sna_file_tex }\OperatorTok{=} \StringTok{"test_yml_fan"}
\NormalTok{srn_file_tex }\OperatorTok{=}\NormalTok{ srt_file_tex }\OperatorTok{+}\NormalTok{ sna_file_tex }\OperatorTok{+} \StringTok{".yml"}
\CommentTok{# Open new file}
\NormalTok{fl_tex_contents }\OperatorTok{=} \BuiltInTok{open}\NormalTok{(srn_file_tex, }\StringTok{'w'}\NormalTok{)}
\CommentTok{# Write to File}
\NormalTok{fl_tex_contents.write(stf_tex_contents)}
\CommentTok{# print}
\end{Highlighting}
\end{Shaded}

\begin{verbatim}
## 908
\end{verbatim}

\begin{Shaded}
\begin{Highlighting}[]
\NormalTok{fl_tex_contents.close()}
\end{Highlighting}
\end{Shaded}

\hypertarget{select-subset-of-values-by-key}{%
\subsubsection{Select Subset of Values by Key}\label{select-subset-of-values-by-key}}

Load Yaml file created prior, the output is a list of dictionaries:

\begin{Shaded}
\begin{Highlighting}[]
\ImportTok{import}\NormalTok{ yaml }
\ImportTok{import}\NormalTok{ pprint}
\CommentTok{# Open yaml file}
\NormalTok{fl_yaml }\OperatorTok{=} \BuiltInTok{open}\NormalTok{(srn_file_tex)}
\CommentTok{# load yaml }
\NormalTok{ls_dict_yml }\OperatorTok{=}\NormalTok{ yaml.load(fl_yaml, Loader}\OperatorTok{=}\NormalTok{yaml.BaseLoader)}
\CommentTok{# type}
\BuiltInTok{type}\NormalTok{(ls_dict_yml)}
\end{Highlighting}
\end{Shaded}

\begin{verbatim}
## <class 'list'>
\end{verbatim}

\begin{Shaded}
\begin{Highlighting}[]
\BuiltInTok{type}\NormalTok{(ls_dict_yml[}\DecValTok{0}\NormalTok{])}
\CommentTok{# display}
\end{Highlighting}
\end{Shaded}

\begin{verbatim}
## <class 'dict'>
\end{verbatim}

\begin{Shaded}
\begin{Highlighting}[]
\NormalTok{pprint.pprint(ls_dict_yml, width}\OperatorTok{=}\DecValTok{1}\NormalTok{)}
\end{Highlighting}
\end{Shaded}

\begin{verbatim}
## [{'core': [{'code': "c('word1','word2')\n"
##                     'function()\n'
##                     'for '
##                     '(ctr '
##                     'in '
##                     'c(1,2)) '
##                     '{}\n',
##             'package': 'r'},
##            {'code': 'group_by()\n',
##             'package': 'dplyr'}],
##   'date': '2020-05-02',
##   'description': 'Frequency '
##                  'table, '
##                  'bar '
##                  'chart '
##                  'and '
##                  'histogram.\n'
##                  'R '
##                  'function '
##                  'and '
##                  'lapply '
##                  'to '
##                  'generate '
##                  'graphs/tables '
##                  'for '
##                  'different '
##                  'variables.\n',
##   'file': 'matrix_matlab',
##   'output': {'pdf_document': {'includes': {'in_header': '../preamble.tex'},
##                               'pandoc_args': '../_output_kniti_pdf.yaml'}},
##   'title': 'One '
##            'Variable '
##            'Graphs '
##            'and '
##            'Tables',
##   'urlcolor': 'blue'},
##  {'core': [{'code': 'setwd()\n',
##             'package': 'r'},
##            {'code': 'write_csv()\n',
##             'package': 'readr'}],
##   'date': '2020-05-02',
##   'date_start': '2018-12-01',
##   'description': 'Opening '
##                  'a '
##                  'Dataset.\n',
##   'file': 'matrix_algebra_rules',
##   'title': 'Opening '
##            'a '
##            'Dataset',
##   'titleshort': 'Opening '
##                 'a '
##                 'Dataset'},
##  {'description': 'Third '
##                  'file '
##                  'description.',
##   'file': 'matrix_two',
##   'title': 'Third '
##            'file',
##   'titleshort': 'Third '
##                 'file'}]
\end{verbatim}

Select yaml information by \emph{file} name which is a key shared by components of the list:

\begin{Shaded}
\begin{Highlighting}[]
\NormalTok{ls_str_file_ids }\OperatorTok{=}\NormalTok{ [}\StringTok{'matrix_two'}\NormalTok{]}
\NormalTok{ls_dict_selected }\OperatorTok{=}\NormalTok{ [dict_yml }\ControlFlowTok{for}\NormalTok{ dict_yml }\KeywordTok{in}\NormalTok{ ls_dict_yml }\ControlFlowTok{if}\NormalTok{ dict_yml[}\StringTok{'file'}\NormalTok{] }\KeywordTok{in}\NormalTok{ ls_str_file_ids]}
\NormalTok{pprint.pprint(ls_dc_selected, width}\OperatorTok{=}\DecValTok{1}\NormalTok{)}
\end{Highlighting}
\end{Shaded}

\begin{verbatim}
## [{'date': datetime.date(2020, 5, 2),
##   'description': 'Frequency '
##                  'table, '
##                  'bar '
##                  'chart '
##                  'and '
##                  'histogram',
##   'file': 'mat_matlab',
##   'title': 'One '
##            'Variable '
##            'Graphs '
##            'and '
##            'Tables',
##   'val': 1}]
\end{verbatim}

\hypertarget{dump-list-of-dictionary-as-yaml}{%
\subsubsection{Dump List of Dictionary as YAML}\label{dump-list-of-dictionary-as-yaml}}

\begin{itemize}
\tightlist
\item
  \href{https://stackoverflow.com/a/8641732/8280804}{py yaml dump pipe}
\end{itemize}

Given a list of dictionaries, dump values to yaml. Note that dumped output does not use pipe for long sentences, but use single quote and space line, which works with the \href{https://github.com/FanWangEcon/pyfan/blob/master/pyfan/util/rmd/rmdparse.py}{rmdparrse.py} function without problem.

\begin{Shaded}
\begin{Highlighting}[]
\NormalTok{ls_dict_selected }\OperatorTok{=}\NormalTok{ [dict_yml }\ControlFlowTok{for}\NormalTok{ dict_yml }\KeywordTok{in}\NormalTok{ ls_dict_yml }
                    \ControlFlowTok{if}\NormalTok{ dict_yml[}\StringTok{'file'}\NormalTok{] }\KeywordTok{in}\NormalTok{ [}\StringTok{'matrix_two'}\NormalTok{,}\StringTok{'matrix_matlab'}\NormalTok{]]}
\BuiltInTok{print}\NormalTok{(yaml.dump(ls_dict_selected))}
\end{Highlighting}
\end{Shaded}

\begin{verbatim}
## - core:
##   - code: 'c(''word1'',''word2'')
## 
##       function()
## 
##       for (ctr in c(1,2)) {}
## 
##       '
##     package: r
##   - code: 'group_by()
## 
##       '
##     package: dplyr
##   date: '2020-05-02'
##   description: 'Frequency table, bar chart and histogram.
## 
##     R function and lapply to generate graphs/tables for different variables.
## 
##     '
##   file: matrix_matlab
##   output:
##     pdf_document:
##       includes:
##         in_header: ../preamble.tex
##       pandoc_args: ../_output_kniti_pdf.yaml
##   title: One Variable Graphs and Tables
##   urlcolor: blue
## - description: Third file description.
##   file: matrix_two
##   title: Third file
##   titleshort: Third file
\end{verbatim}

\hypertarget{appendix-appendix}{%
\appendix}


\hypertarget{index-and-code-links}{%
\chapter{Index and Code Links}\label{index-and-code-links}}

\hypertarget{array-matrix-dataframe-links}{%
\section{Array, Matrix, Dataframe links}\label{array-matrix-dataframe-links}}

\hypertarget{section-1.1-arrayarray-links}{%
\subsection{\texorpdfstring{\protect\hyperlink{array}{Section 1.1 Array} links}{Section 1.1 Array links}}\label{section-1.1-arrayarray-links}}

\begin{enumerate}
\def\labelenumi{\arabic{enumi}.}
\tightlist
\item
  \href{https://fanwangecon.github.io/pyfan/vig/amto/array/htmlpdfr/fp_ary_string.html}{Python String Manipulation Examples}: \href{https://github.com/FanWangEcon/pyfan/blob/master/vig/amto/array//fp_ary_string.Rmd}{\textbf{rmd}} \textbar{} \href{https://github.com/FanWangEcon/pyfan/blob/master/vig/amto/array/htmlpdfr/fp_ary_string.R}{\textbf{r}} \textbar{} \href{https://github.com/FanWangEcon/pyfan/blob/master/vig/amto/array/htmlpdfr/fp_ary_string.pdf}{\textbf{pdf}} \textbar{} \href{https://fanwangecon.github.io/pyfan/vig/amto/array/htmlpdfr/fp_ary_string.html}{\textbf{html}}

  \begin{itemize}
  \tightlist
  \item
    Various string manipulations
  \item
    \textbf{py}: \emph{zip()}
  \end{itemize}
\end{enumerate}

\hypertarget{section-1.2-dictionarydictionary-links}{%
\subsection{\texorpdfstring{\protect\hyperlink{dictionary-1}{Section 1.2 Dictionary} links}{Section 1.2 Dictionary links}}\label{section-1.2-dictionarydictionary-links}}

\begin{enumerate}
\def\labelenumi{\arabic{enumi}.}
\tightlist
\item
  \href{https://fanwangecon.github.io/pyfan/vig/amto/dict/htmlpdfr/fp_dict.html}{Python Dictionary Exampls and Usages}: \href{https://github.com/FanWangEcon/pyfan/blob/master/vig/amto/dict//fp_dict.Rmd}{\textbf{rmd}} \textbar{} \href{https://github.com/FanWangEcon/pyfan/blob/master/vig/amto/dict/htmlpdfr/fp_dict.R}{\textbf{r}} \textbar{} \href{https://github.com/FanWangEcon/pyfan/blob/master/vig/amto/dict/htmlpdfr/fp_dict.pdf}{\textbf{pdf}} \textbar{} \href{https://fanwangecon.github.io/pyfan/vig/amto/dict/htmlpdfr/fp_dict.html}{\textbf{html}}

  \begin{itemize}
  \tightlist
  \item
    List comprehension with dictionary
  \item
    \textbf{py}: \emph{dc = \{`key': ``name'', `val': 1\}}
  \end{itemize}
\end{enumerate}

\hypertarget{system-and-support-links}{%
\section{System and Support links}\label{system-and-support-links}}

\hypertarget{section-2.1-file-in-and-outfile-in-and-out-links}{%
\subsection{\texorpdfstring{\protect\hyperlink{file-in-and-out}{Section 2.1 File In and Out} links}{Section 2.1 File In and Out links}}\label{section-2.1-file-in-and-outfile-in-and-out-links}}

\begin{enumerate}
\def\labelenumi{\arabic{enumi}.}
\tightlist
\item
  \href{https://fanwangecon.github.io/pyfan/vig/support/inout/htmlpdfr/fp_files.html}{Python Reading and Writing to File Examples}: \href{https://github.com/FanWangEcon/pyfan/blob/master/vig/support/inout//fp_files.Rmd}{\textbf{rmd}} \textbar{} \href{https://github.com/FanWangEcon/pyfan/blob/master/vig/support/inout/htmlpdfr/fp_files.R}{\textbf{r}} \textbar{} \href{https://github.com/FanWangEcon/pyfan/blob/master/vig/support/inout/htmlpdfr/fp_files.pdf}{\textbf{pdf}} \textbar{} \href{https://fanwangecon.github.io/pyfan/vig/support/inout/htmlpdfr/fp_files.html}{\textbf{html}}

  \begin{itemize}
  \tightlist
  \item
    Reading from file and replace strings in file.
  \item
    Convert text file to latex using pandoc and clean.
  \item
    Search for files in several folders with file substring.
  \item
    Get path root, file name, file stem, etc from path.
  \item
    \textbf{py}: \emph{open() + write() + replace() + {[}c for b in {[}{[}1,2{]},{[}2,3{]}{]} for c in b{]}}
  \item
    \textbf{subprocess}: \emph{read()}
  \item
    \textbf{pathlib}: \emph{Path().rglob() + Path().stem}
  \item
    \textbf{os}: \emph{remove() + listdir() + path.isfile() + path.splitdrive() + os.path.splitext() + os.path.split()}
  \end{itemize}
\item
  \href{https://fanwangecon.github.io/pyfan/vig/support/inout/htmlpdfr/fp_folders.html}{Python Directory and Folder Operations}: \href{https://github.com/FanWangEcon/pyfan/blob/master/vig/support/inout//fp_folders.Rmd}{\textbf{rmd}} \textbar{} \href{https://github.com/FanWangEcon/pyfan/blob/master/vig/support/inout/htmlpdfr/fp_folders.R}{\textbf{r}} \textbar{} \href{https://github.com/FanWangEcon/pyfan/blob/master/vig/support/inout/htmlpdfr/fp_folders.pdf}{\textbf{pdf}} \textbar{} \href{https://fanwangecon.github.io/pyfan/vig/support/inout/htmlpdfr/fp_folders.html}{\textbf{html}}

  \begin{itemize}
  \tightlist
  \item
    Generate new folders and files.
  \item
    Generate subfolder recursively.
  \item
    Copying and moving files across folders.
  \item
    Aggregate subfolders into a folder and move.
  \item
    \textbf{py}: \emph{open(srt, `w') + write() + close()}
  \item
    \textbf{os}: \emph{os.listdir() + os.path.join(`/', `c:', `fa', `fb')}
  \item
    \textbf{pathlib}: \emph{Path(srt).mkdir(parents=True, exist\_ok=True) + {[}Path(spn).stem for spn in Path(srt).rglob(st){]}}
  \item
    \textbf{shutil}: \emph{shutil.copyfile(`/fa/fl.txt', `/fb/fl.txt') + shutil.copy2(`/fa/fl.txt', `/fb') + shutil.rmtree(`/fb')}
  \item
    \textbf{distutils}: \emph{dir\_util.copy\_tree(`/fa', `/fb')}
  \end{itemize}
\item
  \href{https://fanwangecon.github.io/pyfan/vig/support/inout/htmlpdfr/fp_yaml.html}{Python Yaml File Parsing}: \href{https://github.com/FanWangEcon/pyfan/blob/master/vig/support/inout//fp_yaml.Rmd}{\textbf{rmd}} \textbar{} \href{https://github.com/FanWangEcon/pyfan/blob/master/vig/support/inout/htmlpdfr/fp_yaml.R}{\textbf{r}} \textbar{} \href{https://github.com/FanWangEcon/pyfan/blob/master/vig/support/inout/htmlpdfr/fp_yaml.pdf}{\textbf{pdf}} \textbar{} \href{https://fanwangecon.github.io/pyfan/vig/support/inout/htmlpdfr/fp_yaml.html}{\textbf{html}}

  \begin{itemize}
  \tightlist
  \item
    Parse and read yaml files.
  \item
    \textbf{yaml}: \emph{load(fl\_yaml, Loader=yaml.BaseLoader) + dump()}
  \item
    \textbf{pprint}: \emph{pprint.pprint(ls\_dict\_yml, width=1)}
  \end{itemize}
\end{enumerate}

  \bibliography{book.bib,packages.bib}

\end{document}
